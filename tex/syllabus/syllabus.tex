\documentclass[11pt]{article}
\usepackage[margin=1in]{geometry}
\usepackage{enumitem}
\usepackage{hyperref}
\usepackage{longtable}
\usepackage{array}
\newcolumntype{L}[1]{>{\raggedright\arraybackslash}p{#1}}

\begin{document}

\begin{center}
    \Large \textbf{MMAE 450: Computational Mechanics II} \\
    \large Spring 2026 \\
    \vspace{0.5em}
    Department of Mechanical, Materials, and Aerospace Engineering \\
    Illinois Institute of Technology
\end{center}

\vspace{1em}

\section*{Instructor Information}
\begin{itemize}
    \item \textbf{Instructor:} Dr. Michael Gosz
    \item \textbf{Office:} 207A Rettaliata Engineering Center
    \item \textbf{Phone:} (312) 567-3198
    \item \textbf{Email:} \texttt{gosz@illinoistech.edu}
    \item \textbf{Office Hours:} TBA
    \item \textbf{Class Meetings:} Tue, Thu (1:50pm--3:05pm), Stuart Building 238
\end{itemize}

\vspace{0.5em}

\section*{Course Information}
\begin{tabular}{ll}
\textbf{Course:} & MMAE 450 \\
\textbf{Title:} & Computational Mechanics II \\
\textbf{Credits:} & 3 credit hours \\
\textbf{Prerequisites:} & MMAE 350
\end{tabular}

\section*{Course Description}
This course builds on MMAE 350 and develops advanced computational
methods for modeling engineering systems governed by partial differential
equations. Emphasis is placed on translating physical assumptions into
governing equations, discretizing those equations using finite
difference, finite volume, and finite element methods, and interpreting numerical results in the context of what the model predicts and what physics permits.

This course emphasizes engineering modeling and numerical
reasoning rather than programming for its own sake. Computational tools are used to support physical insight, model
verification, and engineering design decisions.

A central theme of the course is the integration of physics-based
modeling with data-driven approaches. Students are introduced to regression, classification, and
physics-informed neural networks (PINNs) as tools for prediction and
surrogate modeling that complement and extend traditional numerical
techniques. Students gain experience implementing algorithms
in Python using Jupyter notebooks, validating numerical results, and communicating computational findings
clearly and professionally in an engineering context.

\section*{Learning Objectives}

By the end of this course, students will be able to:

\begin{enumerate}[leftmargin=2em]
    \item Formulate governing equations for heat transfer, wave propagation,
          and transport phenomena in engineering systems.
    \item Discretize partial differential equations using finite difference,
          finite volume, and finite element methods.
    \item Implement and analyze numerical solvers for linear and nonlinear
          systems arising from discretized models.
    \item Evaluate numerical accuracy, stability, conditioning, and convergence
          of computational methods.
    \item Apply weak formulations and finite element approximations to
          one- and two-dimensional problems.
    \item Develop regression and classification models for engineering datasets
          and interpret model parameters.
    \item Construct and train physics-informed neural networks (PINNs) for
          boundary value problems.
    \item Compare physics-based, data-driven, and hybrid modeling approaches,
          identifying advantages and limitations.
    \item Implement reproducible computational workflows using Python,
          Jupyter notebooks, and version-controlled resources.
    \item Deploy computational and machine learning models using modern
          cloud-based tools.
\end{enumerate}

\section*{Weekly Topics}

The course is organized around a progression from governing equations,
to numerical discretization, to modern data-driven and hybrid modeling
approaches. The following outline indicates the primary topics covered
during the semester; exact pacing may be adjusted as needed.

\begin{enumerate}[leftmargin=2em]
    \item Review of nonlinear equations, matrix systems, and classification
          of partial differential equations (elliptic, parabolic, hyperbolic).
    \item Time-dependent partial differential equations:
          formulation and physical interpretation.
    \item Finite difference methods for transient heat conduction
          in one and two dimensions.
    \item Wave propagation problems in solids:
          numerical discretization and stability considerations.
    \item Stability, conditioning, and convergence analysis of numerical
          schemes (including von Neumann analysis).
    \item Conservation laws and flux formulations:
          introduction to the finite volume method.
    \item Finite volume methods for advection--diffusion problems,
          including numerical diffusion and upwinding.
    \item Weak formulations and variational principles for
          boundary value problems.
    \item Finite element methods for heat conduction:
          shape functions, assembly, and solution of global systems.
    \item Finite element modeling of transport and coupled problems
          in one and two dimensions.
    \item Introduction to data-driven modeling in engineering:
          regression and classification methods.
    \item Physics-informed machine learning:
          physics-informed neural networks (PINNs) for boundary value problems.
    \item Surrogate modeling and hybrid approaches:
          combining numerical simulation with machine learning.
    \item Reproducible and cloud-based computational workflows
          for engineering analysis and model deployment.
    \item Integrated computational project development and presentations.
\end{enumerate}

\section*{Assignments \& Assessment}

Student performance in this course will be evaluated using the following
components:

\begin{itemize}[leftmargin=2em]

  \item \textbf{Homework Assignments (35\%)} \\
  Regular problem sets emphasizing advanced numerical methods,
  implementation of algorithms in Jupyter notebooks, and interpretation
  of computational results for engineering models governed by partial
  differential equations.

  \item \textbf{Midterm Exams (30\% total)} \\
  Two in-class midterm exams, spaced throughout the semester, assessing
  conceptual understanding, mathematical formulation, and analysis of
  numerical methods and computational models covered up to each exam.

  \item \textbf{Final Exam (20\%)} \\
  A comprehensive exam evaluating students’ ability to integrate
  numerical methods, physical modeling, and computational reasoning
  across the full scope of the course.

  \item \textbf{Final Project (15\%)} \\
  A focused computational project in which students develop and analyze
  a computational mechanics model. Projects may emphasize physics-based
  simulation, data-driven modeling, or hybrid approaches that integrate
  numerical methods with machine learning. Results will be documented in
  a written report, with selected projects presented orally.

\end{itemize}

\noindent
Homework assignments consist primarily of weekly coding exercises in
Jupyter notebooks. Midterm exams include closed-book theoretical
components and open-book computational components. The final project is
a team-based effort and may focus on either a mechanics-based numerical model or a machine learning–based surrogate model.

\section*{Tools \& Resources}
\begin{itemize}[leftmargin=2em]
    \item \textbf{Primary Computational Environment:} \\
    Python is the primary programming language used in this course.
    Students will work extensively with Jupyter notebooks and standard
    scientific libraries, including NumPy, SciPy, SymPy, Matplotlib,
    and scikit-learn. Selected examples will also use PyTorch for
    physics-informed machine learning applications.
    \item \textbf{Software Setup:} \\
    All students are expected to work in a local Python virtual
    environment, following the same setup procedure introduced in
    MMAE~350. This ensures reproducibility and consistency across
    computational assignments and projects.
\item \textbf{Textbook (Primary Reference):} \\
\emph{Computational Mechanics: A Modern Introduction with Machine Learning
and AWS Workflows} \\
by M.~Gosz. This text serves as the primary reference for the course and
provides the theoretical foundations, numerical methods, and
computational examples used throughout the semester.
    \item \textbf{Companion Code Repository:} \\
	A public GitHub repository accompanies the textbook and contains
	Jupyter notebooks, example codes, datasets, and a
	\texttt{requirements.txt} file. All computational assignments in the
	course build directly on these materials. The repository is available at:
	\begin{center}
	\url{https://github.com/gosz450/computational-mechanics-companion}
	\end{center}
    \item \textbf{Cloud-Based Workflows:} \\
    Selected modules and the final project may involve cloud-based
    execution and deployment of computational or machine learning models.
    These activities will build on local workflows and will be introduced
    incrementally during the semester.
    \item \textbf{Supplementary References (Optional):}
    \begin{itemize}[noitemsep]
        \item Chapra \& Canale, \emph{Numerical Methods for Engineers}
        \item Ferziger \& Perić, \emph{Computational Fluid Dynamics}
        \item Géron, \emph{Hands-On Machine Learning with Scikit-Learn, Keras \& TensorFlow}
    \end{itemize}

\end{itemize}

\section*{Course Calendar --- Spring 2026}
\renewcommand{\arraystretch}{1.15}
\emph{Note: The course calendar is subject to minor adjustments based on class progress.}
\begin{longtable}{|L{2.2cm}|L{1.6cm}|L{10cm}|}
\hline
\textbf{Date} & \textbf{Day} & \textbf{Topic} \\ \hline
\endfirsthead

\hline
\textbf{Date} & \textbf{Day} & \textbf{Topic} \\ \hline
\endhead

Jan 13 & Tue & Course introduction; computational modeling mindset; course tools and workflow \\
Jan 15 & Thu & Review of nonlinear systems; PDE classification and modeling assumptions \\
Jan 20 & Tue & Time-dependent PDEs: physical interpretation and formulation \\
Jan 22 & Thu & Finite difference methods for 1D transient heat conduction \\
Jan 27 & Tue & Finite difference methods for 2D heat conduction \\
Jan 29 & Thu & Numerical wave propagation problems; stability considerations \\
\hline
Feb 3  & Tue & Stability and conditioning of numerical schemes; von Neumann analysis \\
Feb 5  & Thu & Convergence, error estimation, and grid refinement studies \\
Feb 10 & Tue & Applications in transient heat conduction (1D/2D) \\
Feb 12 & Thu & Case study: fin heat transfer and validation of numerical models \\
Feb 17 & Tue & Review and problem session \\
Feb 19 & Thu & \textbf{Midterm Exam 1} \\
\hline
Feb 24 & Tue & Conservation laws and flux formulations \\
Feb 26 & Thu & Finite volume methods for advection--diffusion problems \\
Mar 3  & Tue & Numerical diffusion, upwinding, and stability in finite volume schemes \\
Mar 5  & Thu & Weak formulations and variational principles for boundary value problems \\
Mar 10 & Tue & Finite element method for heat conduction: shape functions and assembly \\
Mar 12 & Thu & Finite element examples and verification \\
Mar 17 & Tue & \textbf{Spring Break} \\
Mar 19 & Thu & \textbf{Spring Break} \\
\hline
Mar 24 & Tue & Review and problem session \\
Mar 26 & Thu & \textbf{Midterm Exam 2} \\
Mar 31 & Tue & Introduction to data-driven modeling: regression and classification \\
\hline
Apr 2  & Thu & Regression models for mechanics data; interpretation and limitations \\
Apr 7  & Tue & Physics-informed machine learning: PINNs for boundary value problems \\
Apr 9  & Thu & Surrogate modeling: hybrid physics-based and ML approaches \\
Apr 14 & Tue & Cloud-based workflows for computational mechanics and ML models \\
Apr 16 & Thu & Project workshop: model selection and scope definition \\
Apr 21 & Tue & Advanced applications and case studies \\
Apr 23 & Thu & Review and integration: numerical methods, FEM, and ML \\
Apr 28 & Tue & Final review \\
Apr 30 & Thu & \textbf{Final Exam} \\
May 5  & Tue & Final project presentations (selected projects) \\
May 7  & Thu & Final project presentations (selected projects) \\
\hline
\end{longtable}

% --------------------------------------------------
\section*{Course Policies}

\subsection*{Attendance}
Regular attendance is expected. Students are responsible for all material covered in class and all announcements made, regardless of attendance. Excessive unexcused absences may negatively impact your performance and grade.

\subsection*{Late Work}
Assignments are due on the date and time specified. Late work will generally not be accepted unless prior arrangements are made with the instructor, or unless extraordinary circumstances can be documented. In such cases, partial credit may be given at the discretion of the instructor.

\subsection*{Academic Integrity}
Illinois Tech expects all students to uphold the highest standards of academic honesty. Cheating, plagiarism, or any other form of academic dishonesty will not be tolerated. Violations will be reported and may result in failure of the assignment, failure of the course, and/or additional disciplinary action as outlined in the Illinois Tech Code of Conduct. For more information, see: \url{https://www.iit.edu/student-affairs/student-handbook}

\subsection*{Collaboration}
Collaboration on homework assignments is permitted at the level of discussing concepts and approaches. However, all work turned in must be your own. Copying code, solutions, or written responses from another student or from online sources constitutes academic dishonesty.

\subsection*{Use of Technology}
Students are encouraged to use Python, and other computational tools as required by the course. Any use of technology during quizzes or exams must be explicitly permitted by the instructor. Unauthorized use of technology during exams will be considered a violation of academic integrity.

\subsection*{Communication}
Course announcements will be made in class and via email or the course LMS (Canvas). It is your responsibility to check email and the course site regularly. Email is the preferred method of communication outside of class hours.

\subsection*{Professional Conduct}
Respectful behavior is expected in class, in labs, and in all course-related activities. Disruptive conduct will not be tolerated. Students should contribute to a learning environment that supports diversity of thought and experience.

% --------------------------------------------------
\section*{Accessibility Statement}

Illinois Tech is committed to providing an inclusive educational environment and making every effort to ensure equal access for all students. If you are a student with a documented disability and require reasonable academic accommodations, please contact the Center for Disability Resources (CDR) as soon as possible. Accommodations are determined on a case-by-case basis, considering the student’s documented needs and the technical requirements of the course.

To request accommodations or learn more, contact:
\begin{itemize}
  \item \textbf{Center for Disability Resources (CDR)} \\
    Phone: 312-567-5744 \\
    Email: \texttt{disabilities@illinoistech.edu}
\end{itemize}

Students must submit documentation and meet with CDR to establish eligibility and receive an accommodation letter. Provide your letter to the instructor early in the semester to discuss appropriate implementation of accommodations. 

\noindent For more information, visit: \url{https://www.iit.edu/cdr}

% --------------------------------------------------
\pagebreak
\section*{University Academic Calendar — Spring 2026}
\renewcommand{\arraystretch}{1.2}
\begin{longtable}{|>{\raggedright\arraybackslash}p{4cm}|>{\raggedright\arraybackslash}p{10cm}|}
\hline
\textbf{Date} & \textbf{Event} \\ \hline
January 12 & Spring Courses Begin \\
January 19 & Martin Luther King, Jr. Day — No Classes \\
January 20 & Last Day to Add/Drop for Full Semester Courses with No Tuition Charges \\
January 27 & Last Day to Request Late Registration \\
March 13 & Midterm Grades Due \\
March 16--21 & Spring Break Week — No Classes \\
March 30 & Last Day to Withdraw for Full Semester Courses \\
April 6 & Fall Registration Begins \\
May 2 & Last Day of Spring Courses \\
May 3 & Last Day to Request an Incomplete Grade \\
May 4--9 & Final Exam Week (Final Grading Begins May 4) \\
May 13 & Final Grades Due at Noon (CST) \\
\hline
\end{longtable}

\end{document}