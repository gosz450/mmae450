\documentclass[11pt]{article}

\usepackage[margin=1in]{geometry}
\usepackage{amsmath,amssymb}
\usepackage{bm}
\usepackage{enumitem}
\usepackage{hyperref}

\setlength{\parskip}{0.75em}
\setlength{\parindent}{0pt}

\title{\textbf{MMAE 450 — Homework 2}\\
Newton’s Method for Nonlinear Systems}
\author{}
\date{Due: January 27, 2026}

\begin{document}
\maketitle

\section*{Objectives}

The objectives of this assignment are to:
\begin{itemize}[leftmargin=2em]
  \item Introduce Newton’s method for solving systems of nonlinear equations.
  \item Reinforce residual-based formulations of governing equations.
  \item Practice computing Jacobian matrices analytically or symbolically.
  \item Implement a Newton solver using numerical linear algebra.
\end{itemize}

This homework emphasizes algorithmic structure and physical interpretation rather than algebraic complexity.

\section*{Problem: Steady-State Nonlinear Heat Balance}

Consider two thermal nodes with unknown temperatures $T_1$ and $T_2$.
The nodes exchange heat by linear conduction, and node 1 loses heat to the
environment by thermal radiation.

The steady-state governing equations are given by
\begin{align}
R_1(T_1,T_2) &= k\,(T_2 - T_1) + \sigma\,(T_1^4 - T_\infty^4), \label{eq:R1} \\
R_2(T_1,T_2) &= k\,(T_1 - T_2). \label{eq:R2}
\end{align}

Here:
\begin{itemize}[leftmargin=2em]
  \item $k > 0$ is a conduction coefficient,
  \item $\sigma > 0$ is a radiation coefficient,
  \item $T_\infty$ is the ambient temperature.
\end{itemize}

\subsection*{(a) Residual Formulation}

Write the system \eqref{eq:R1}–\eqref{eq:R2} in vector residual form,
\[
\bm{R}(\bm{T}) =
\begin{bmatrix}
R_1(T_1,T_2) \\
R_2(T_1,T_2)
\end{bmatrix}
=
\bm{0},
\qquad
\bm{T} =
\begin{bmatrix}
T_1 \\
T_2
\end{bmatrix}.
\]

Clearly identify the unknown vector and the residual components.

\subsection*{(b) Jacobian Matrix}

Compute the Jacobian matrix
\[
\bm{J}(\bm{T}) = \frac{\partial \bm{R}}{\partial \bm{T}}.
\]

You may compute the Jacobian:
\begin{itemize}[leftmargin=2em]
  \item analytically, or
  \item using a symbolic tool such as \texttt{SymPy}.
\end{itemize}

\subsection*{(c) Newton Iteration}

Write the Newton linear system solved at iteration $k$ and the update equation
for $\bm{T}^{(k+1)}$.

\subsection*{(d) Numerical Solution}

Implement Newton’s method in Python using \textbf{NumPy only} (no symbolic
operations inside the solver).

Your implementation must:
\begin{itemize}[leftmargin=2em]
  \item use a relative residual convergence criterion
  \[
  \frac{\|\bm{R}(\bm{T}^{(k)})\|}{\|\bm{R}(\bm{T}^{(0)})\|} < \varepsilon,
  \]
  where $\varepsilon$ is a user-prescribed tolerance,
  \item report the converged solution $(T_1,T_2)$,
  \item report the number of Newton iterations required.
\end{itemize}

You may choose reasonable numerical values for $k$, $\sigma$, and $T_\infty$.

\subsection*{(e) Convergence Behavior}

Plot the relative residual norm versus Newton iteration number and comment briefly
on the observed convergence behavior.

\section*{Submission Instructions}

Submit a single Jupyter notebook (\texttt{.ipynb}) containing:
\begin{itemize}[leftmargin=2em]
  \item a clear problem setup,
  \item residual and Jacobian definitions,
  \item the Newton solver implementation,
  \item results and discussion.
\end{itemize}

Ensure that your notebook is well-organized and clearly documented.



\end{document}