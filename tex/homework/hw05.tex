%=======================================================================
% MMAE 450
% Homework: Finite Volume Method for 1D Advection--Diffusion
% Total: 20 points
%=======================================================================

\documentclass[11pt]{article}

\usepackage[margin=1in]{geometry}
\usepackage{amsmath,amssymb}
\usepackage{graphicx}
\usepackage{enumitem}
\usepackage{hyperref}

\setlength{\parindent}{0pt}
\setlength{\parskip}{6pt}

\begin{document}

%=======================================================================
\begin{center}
{\Large \textbf{MMAE 450 -- Computational Mechanics II}} \\[6pt]
{\large Homework 5: Finite Volume Method for 1D Advection--Diffusion} \\[6pt]
\textbf{Total: 20 points} \\
\textbf{Due: March 13, 11:59 pm}
\end{center}
%=======================================================================

\hrule
\vspace{1em}

\textbf{Objective.}
In this assignment you will implement the finite volume method (FVM)
for the one-dimensional advection--diffusion equation and investigate
the influence of the cell Peclet number on numerical behavior.

\vspace{1em}
\hrule
\vspace{1em}

%=======================================================================
\section*{Problem Description}

A contaminant is introduced into a straight river segment of length
$L = 1000$ m. The river has a uniform downstream velocity $v$
and a constant dispersion (diffusion) coefficient $\alpha$.

The pollutant concentration $T(x,t)$ satisfies the
one-dimensional advection--diffusion equation

\[
\frac{\partial T}{\partial t}
+
v \frac{\partial T}{\partial x}
=
\alpha \frac{\partial^2 T}{\partial x^2},
\qquad 0 < x < L.
\]

Use

\[
\alpha = 0.5 \text{ m}^2/\text{s}.
\]

\vspace{0.5em}

\textbf{Initial condition (localized spill):}

\[
T(x,0) =
\begin{cases}
1, & 0.4L \le x \le 0.6L, \\
0, & \text{otherwise}.
\end{cases}
\]

\vspace{0.5em}

\textbf{Boundary conditions:}

\begin{itemize}
\item Upstream boundary (inlet): $T(0,t) = 0$
\item Downstream boundary (open outlet):
\[
\frac{\partial T}{\partial x}(L,t) = 0
\]
\end{itemize}

Assume the flow is steady and the pollutant does not affect the velocity field.

\vspace{1em}
\hrule
\vspace{1em}

%=======================================================================
\section*{(a) Derivation (5 pts)}

Starting from the conservative form

\[
\frac{\partial T}{\partial t}
+
\frac{\partial}{\partial x}
\left(
vT - \alpha \frac{\partial T}{\partial x}
\right)
=
0,
\]

derive the fully discrete finite volume update using:

\begin{itemize}
\item Central interpolation for the convective flux
\item Central difference for the diffusive flux
\item Forward Euler in time
\end{itemize}

Present the final update equation clearly.

\vspace{1em}
\hrule
\vspace{1em}

%=======================================================================
\section*{(b) Implementation (10 pts)}

Write a Python program that:

\begin{enumerate}
\item Discretizes the domain into $N=100$ uniform cells.
\item Implements the fully discrete update derived in part (a).
\item Applies the boundary conditions correctly at each time step.
\item Advances the solution to $t = 200$ s.
\end{enumerate}

Plot the solution at

\[
t = 0,\quad 50,\quad 100,\quad 200 \text{ s}.
\]

Plots must include labeled axes and a legend.

\vspace{1em}
\hrule
\vspace{1em}

%=======================================================================
\section*{(c) Peclet Number Study (5 pts)}

Repeat the simulation for

\[
v = 1,\quad 5,\quad 10 \text{ m/s}.
\]

For each case:

\begin{enumerate}
\item Compute the cell Peclet number
\[
\mathrm{Pe}_{\Delta x} = \frac{v \Delta x}{\alpha}.
\]
\item Plot the solution at $t=200$ s.
\item Comment briefly on oscillations or instability.
\end{enumerate}

Relate your observations to the magnitude of the Peclet number.

\vspace{1em}
\hrule
\vspace{1em}

%=======================================================================

%=======================================================================
\section*{Submission Requirements}

Submit:

\begin{itemize}
\item A single PDF containing derivations, plots, and discussion.
\item Your Python code.
\end{itemize}

Code must be readable and commented.

\vspace{2em}

\begin{center}
\textbf{Total: 20 points}
\end{center}

\end{document}