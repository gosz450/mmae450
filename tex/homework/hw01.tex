\documentclass[11pt]{article}

\usepackage[margin=1in]{geometry}
\usepackage{amsmath,amssymb}
\usepackage{enumitem}
\usepackage{hyperref}

\setlength{\parindent}{0pt}
\setlength{\parskip}{0.75em}

\begin{document}

\begin{center}
  {\Large \textbf{MMAE 450 — Computational Mechanics}}\\
  {\large \textbf{Homework 1: Getting Started with Computational Thinking}}\\
  {Due Tuesday 1/20/2026 11:59pm}\\
  {10 Points}\\
  \vspace{0.5em}
  Spring 2026\\
  M. Gosz
\end{center}

\vspace{1em}

\textbf{Purpose.}
This first homework is intentionally low-stakes. The goal is to confirm your
computational workflow, practice running a Jupyter notebook, and begin thinking
about how numerical computation supports engineering modeling and interpretation.

\vspace{0.5em}

\textbf{What to Submit.}
Submit a single PDF containing your written responses to Problems 1--3.
You may also submit your edited notebook file
(\texttt{Notebook~01}) as a separate upload (recommended, but not required).

\vspace{0.5em}

\textbf{Clarity and interpretation matter.} Full sentences are expected where
requested.

\vspace{1em}

% ------------------------------------------------------------
\section*{Problem 1 — What Is Computational Mechanics?}

In your own words (approximately 5--8 sentences total), answer the following:

\begin{enumerate}[label=(\alph*)]
  \item What is computational mechanics?
  \item Where does computation enter the engineering modeling process?
  \item Why is interpretation or validation necessary even when code runs
        without errors?
\end{enumerate}

Your response should emphasize concepts rather than specific software tools.

\vspace{1em}

% ------------------------------------------------------------
\section*{Problem 2 — Running and Modifying Notebook 01}

Open \textbf{Notebook 01 — Python as a Computational Tool} and complete the
following tasks.

\begin{enumerate}[label=(\alph*)]
  \item In the ``Scalars'' section of the notebook, set
  \[
    L = 3.0, \qquad q_0 = 1500.0.
  \]
  Report the resulting value of \texttt{total\_load}.

  \item In the plotting section, change the function definition from
  \[
    y = \sin(2\pi x)
  \]
  to
  \[
    y = x^2.
  \]
  Include the updated plot in your PDF (a screenshot is acceptable).

  \item In 2--3 sentences, describe how and why the plot changed when you
  modified the function.

  \item Modify the array definition so that \texttt{x} contains
  \textbf{100 equally spaced points} between 0 and 1. View the plots again
  and include the updated plot in your PDF.
  In 2--3 sentences, describe how increasing the number of sampling points
  affects the appearance and interpretation of the plot.
\end{enumerate}

% ------------------------------------------------------------
\section*{Problem 3 — Interpretation}

Consider the linear system solved in Notebook~01:
\[
A\mathbf{x} = \mathbf{b}.
\]

In approximately 3--5 sentences, address the following:

\begin{itemize}
  \item What does it mean to ``solve'' a linear system numerically?
  \item What quantity is being computed?
  \item Why do linear systems appear so frequently in computational mechanics?
\end{itemize}

Focus on interpretation rather than mathematical formalism.

\vspace{1em}

% ------------------------------------------------------------
\section*{Submission Notes}

\begin{itemize}
  \item Your PDF may be typed or neatly handwritten.
  \item Clearly label each problem and subpart.
  \item This assignment emphasizes understanding and interpretation rather than
        technical difficulty.
\end{itemize}

\vspace{1em}

\textbf{Reminder.}
Struggling slightly with new tools is normal. The goal is steady progress and
developing intuition, not perfection.

\end{document}