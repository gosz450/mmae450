% -------------------------------------------------
% Packages
% -------------------------------------------------
\usepackage{amsmath, amsfonts}
\usepackage{graphicx}
\usepackage{booktabs}
\usepackage{hyperref}
\usepackage{bm}
\usepackage{siunitx}
\usepackage{listings}

% Figure search paths (relative to tex/lectures/)
\graphicspath{{../figures/lectures/}{../figures/shared/}}

% -------------------------------------------------
% Notes (speaker notes)
% -------------------------------------------------
\usepackage{pgfpages}
% Uncomment ONE of these for speaker notes:
% \setbeameroption{show notes} % notes only (for printing notes)
% \setbeameroption{show notes on second screen=right} % slides + notes

% -------------------------------------------------
% TikZ
% -------------------------------------------------
\usepackage{tikz}
\usetikzlibrary{matrix, calc}
\usepackage{xcolor} % (tikz loads xcolor, but explicit is fine)
\usetikzlibrary{arrows.meta, decorations.pathreplacing}


% -------------------------------------------------
% Themes
%--------------------------------------------------
\usetheme{default}
\usecolortheme{default}
\setbeamertemplate{navigation symbols}{}

% -----------------------------
% Code formatting
% -----------------------------
\definecolor{codegray}{RGB}{245,245,245}

\lstset{
  backgroundcolor=\color{codegray},
  basicstyle=\ttfamily\small,
  frame=single,
  breaklines=true,
  showstringspaces=false
}

% --- Shortcuts ---
%\newcommand{\dd}{\,\mathrm{d}}
\newcommand{\Grad}{\nabla}
\newcommand{\Div}{\nabla\!\cdot}
\newcommand{\bx}{\bm{x}}
\newcommand{\bn}{\bm{n}}
\newcommand{\bq}{\bm{q}}

%==========================
% Flipbook macro (put in preamble or before frames)
%==========================
\newcommand{\GSFlipFrame}[2]{%
\begin{frame}{Gauss--Seidel sweep (current point $(#1,#2)$)}
\centering
\hspace{3.5cm}
\vspace{3cm}
\begin{tikzpicture}[scale=2.2, every node/.style={font=\small}]
  % -----------------------------
  % USER SETTINGS
  % -----------------------------
  \def\Nx{6}   % i = 0..Nx-1
  \def\Ny{4}   % j = 0..Ny-1

  % current stencil location (interior)
  \def\ci{#1}
  \def\cj{#2}

  % Point spacing
  \def\s{0.85}

  % -----------------------------
  % STYLES
  % -----------------------------
  \tikzset{
    gpG/.style={circle, fill=green!70!black, inner sep=1.1pt},
    gpR/.style={circle, fill=red!75!black,   inner sep=1.1pt},
    gpbG/.style={circle, fill=green!70!black, inner sep=1.5pt},
    gpbR/.style={circle, fill=red!75!black,   inner sep=1.5pt},
    gpc/.style={circle, fill=blue, inner sep=1.7pt},
    stline/.style={dashed, thick},
    labij/.style={font=\normalsize},
    labp/.style={font=\scriptsize, text=gray!70},
  }

  % -----------------------------
  % OUTER RECTANGLE (domain)
  % -----------------------------
  \draw[thick] (0,0) rectangle ({(\Nx-1)*\s},{(\Ny-1)*\s});

  % -----------------------------
  % GRID POINTS + LABELS
  % -----------------------------
  \foreach \j in {0,...,\numexpr\Ny-1\relax} {
    \foreach \i in {0,...,\numexpr\Nx-1\relax} {

      \pgfmathsetmacro{\x}{\i*\s}
      \pgfmathsetmacro{\y}{\j*\s}

      % global index p = i + j*Nx (as in your first figure)
      \pgfmathtruncatemacro{\p}{\i + \j*\Nx}

      % boundary?
      \pgfmathtruncatemacro{\isB}{ (\i==0) || (\i==\Nx-1) || (\j==0) || (\j==\Ny-1) }
      % current center?
      \pgfmathtruncatemacro{\isC}{ (\i==\ci) && (\j==\cj) }

      % "already updated" (GS ordering): j<cj OR (j==cj and i<ci)
      \pgfmathtruncatemacro{\isUpdated}{ (\j<\cj) || ((\j==\cj) && (\i<\ci)) }

      % FORCE left boundary (i==0) and bottom boundary (j==0) to be green
      \pgfmathtruncatemacro{\forceGreen}{ \isB }

      % decide color state:
      % - center: blue
      % - else green if forced green OR updated
      % - else red
      \ifnum\isC=1
        \node[gpc] (P\i\j) at (\x,\y) {};
      \else
        \pgfmathtruncatemacro{\isGreen}{ (\forceGreen==1) || (\isUpdated==1) }
        \ifnum\isB=1
          \ifnum\isGreen=1
            \node[gpbG] (P\i\j) at (\x,\y) {};
          \else
            \node[gpbR] (P\i\j) at (\x,\y) {};
          \fi
        \else
          \ifnum\isGreen=1
            \node[gpG] (P\i\j) at (\x,\y) {};
          \else
            \node[gpR] (P\i\j) at (\x,\y) {};
          \fi
        \fi
      \fi

      % Labels (i,j) above-right; p below-right
      \node[labij, anchor=south west] at (\x+0.05,\y+0.05) {$(\i,\j)$};
      \node[labp,  anchor=north west] at (\x+0.05,\y-0.05) {$\p$};
    }
  }

  % -----------------------------
  % 5-POINT STENCIL (dashed)
  % -----------------------------
  \pgfmathsetmacro{\xc}{\ci*\s}
  \pgfmathsetmacro{\yc}{\cj*\s}

  \draw[stline] (\xc,\yc) -- ({(\ci+1)*\s},\yc);
  \draw[stline] (\xc,\yc) -- ({(\ci-1)*\s},\yc);
  \draw[stline] (\xc,\yc) -- (\xc,{(\cj+1)*\s});
  \draw[stline] (\xc,\yc) -- (\xc,{(\cj-1)*\s});

  % Mapping note
  %\node[font=\scriptsize, align=left] at ({(1)*\s+1.65},{(0)*\s-0.55})
  %{$p = i + j\,N_x$};
\end{tikzpicture}
\end{frame}%
}









