
\documentclass[aspectratio=169]{beamer}

% --- Theme / packages ---
% -------------------------------------------------
% Packages
% -------------------------------------------------
\usepackage{amsmath, amsfonts}
\usepackage{graphicx}
\usepackage{booktabs}
\usepackage{hyperref}
\usepackage{bm}
\usepackage{siunitx}
\usepackage{listings}

% Figure search paths (relative to tex/lectures/)
\graphicspath{{../figures/lectures/}{../figures/shared/}}

% -------------------------------------------------
% Notes (speaker notes)
% -------------------------------------------------
\usepackage{pgfpages}
% Uncomment ONE of these for speaker notes:
% \setbeameroption{show notes} % notes only (for printing notes)
% \setbeameroption{show notes on second screen=right} % slides + notes

% -------------------------------------------------
% TikZ
% -------------------------------------------------
\usepackage{tikz}
\usetikzlibrary{matrix, calc}
\usepackage{xcolor} % (tikz loads xcolor, but explicit is fine)
\usetikzlibrary{arrows.meta, decorations.pathreplacing}


% -------------------------------------------------
% Themes
%--------------------------------------------------
\usetheme{default}
\usecolortheme{default}
\setbeamertemplate{navigation symbols}{}

% -----------------------------
% Code formatting
% -----------------------------
\definecolor{codegray}{RGB}{245,245,245}

\lstset{
  backgroundcolor=\color{codegray},
  basicstyle=\ttfamily\small,
  frame=single,
  breaklines=true,
  showstringspaces=false
}

% --- Shortcuts ---
%\newcommand{\dd}{\,\mathrm{d}}
\newcommand{\Grad}{\nabla}
\newcommand{\Div}{\nabla\!\cdot}
\newcommand{\bx}{\bm{x}}
\newcommand{\bn}{\bm{n}}
\newcommand{\bq}{\bm{q}}

%==========================
% Flipbook macro (put in preamble or before frames)
%==========================
\newcommand{\GSFlipFrame}[2]{%
\begin{frame}{Gauss--Seidel sweep (current point $(#1,#2)$)}
\centering
\hspace{3.5cm}
\vspace{3cm}
\begin{tikzpicture}[scale=2.2, every node/.style={font=\small}]
  % -----------------------------
  % USER SETTINGS
  % -----------------------------
  \def\Nx{6}   % i = 0..Nx-1
  \def\Ny{4}   % j = 0..Ny-1

  % current stencil location (interior)
  \def\ci{#1}
  \def\cj{#2}

  % Point spacing
  \def\s{0.85}

  % -----------------------------
  % STYLES
  % -----------------------------
  \tikzset{
    gpG/.style={circle, fill=green!70!black, inner sep=1.1pt},
    gpR/.style={circle, fill=red!75!black,   inner sep=1.1pt},
    gpbG/.style={circle, fill=green!70!black, inner sep=1.5pt},
    gpbR/.style={circle, fill=red!75!black,   inner sep=1.5pt},
    gpc/.style={circle, fill=blue, inner sep=1.7pt},
    stline/.style={dashed, thick},
    labij/.style={font=\normalsize},
    labp/.style={font=\scriptsize, text=gray!70},
  }

  % -----------------------------
  % OUTER RECTANGLE (domain)
  % -----------------------------
  \draw[thick] (0,0) rectangle ({(\Nx-1)*\s},{(\Ny-1)*\s});

  % -----------------------------
  % GRID POINTS + LABELS
  % -----------------------------
  \foreach \j in {0,...,\numexpr\Ny-1\relax} {
    \foreach \i in {0,...,\numexpr\Nx-1\relax} {

      \pgfmathsetmacro{\x}{\i*\s}
      \pgfmathsetmacro{\y}{\j*\s}

      % global index p = i + j*Nx (as in your first figure)
      \pgfmathtruncatemacro{\p}{\i + \j*\Nx}

      % boundary?
      \pgfmathtruncatemacro{\isB}{ (\i==0) || (\i==\Nx-1) || (\j==0) || (\j==\Ny-1) }
      % current center?
      \pgfmathtruncatemacro{\isC}{ (\i==\ci) && (\j==\cj) }

      % "already updated" (GS ordering): j<cj OR (j==cj and i<ci)
      \pgfmathtruncatemacro{\isUpdated}{ (\j<\cj) || ((\j==\cj) && (\i<\ci)) }

      % FORCE left boundary (i==0) and bottom boundary (j==0) to be green
      \pgfmathtruncatemacro{\forceGreen}{ \isB }

      % decide color state:
      % - center: blue
      % - else green if forced green OR updated
      % - else red
      \ifnum\isC=1
        \node[gpc] (P\i\j) at (\x,\y) {};
      \else
        \pgfmathtruncatemacro{\isGreen}{ (\forceGreen==1) || (\isUpdated==1) }
        \ifnum\isB=1
          \ifnum\isGreen=1
            \node[gpbG] (P\i\j) at (\x,\y) {};
          \else
            \node[gpbR] (P\i\j) at (\x,\y) {};
          \fi
        \else
          \ifnum\isGreen=1
            \node[gpG] (P\i\j) at (\x,\y) {};
          \else
            \node[gpR] (P\i\j) at (\x,\y) {};
          \fi
        \fi
      \fi

      % Labels (i,j) above-right; p below-right
      \node[labij, anchor=south west] at (\x+0.05,\y+0.05) {$(\i,\j)$};
      \node[labp,  anchor=north west] at (\x+0.05,\y-0.05) {$\p$};
    }
  }

  % -----------------------------
  % 5-POINT STENCIL (dashed)
  % -----------------------------
  \pgfmathsetmacro{\xc}{\ci*\s}
  \pgfmathsetmacro{\yc}{\cj*\s}

  \draw[stline] (\xc,\yc) -- ({(\ci+1)*\s},\yc);
  \draw[stline] (\xc,\yc) -- ({(\ci-1)*\s},\yc);
  \draw[stline] (\xc,\yc) -- (\xc,{(\cj+1)*\s});
  \draw[stline] (\xc,\yc) -- (\xc,{(\cj-1)*\s});

  % Mapping note
  %\node[font=\scriptsize, align=left] at ({(1)*\s+1.65},{(0)*\s-0.55})
  %{$p = i + j\,N_x$};
\end{tikzpicture}
\end{frame}%
}










\usepackage{tikz}

\usetikzlibrary{arrows.meta,decorations.pathmorphing}


\title[MMAE 450]{Heat Conduction: From Balance Law to FTCS}
\subtitle{(Boundary Conditions)}
\author{Mike Gosz}
\institute{MMAE 450}
\date{}

\begin{document}

% -------------------------
\begin{frame}
  \titlepage
\end{frame}
% ------------------------------------------------------------
% Slides: Neumann (flux) BC at x=0 for 1D heat equation (Cartesian)
% ------------------------------------------------------------

\begin{frame}{1D Heat Equation (Cartesian) and a Prescribed Flux at $x=0$}
We consider the 1D heat equation on $0<x<L$:
\[
\frac{\partial T}{\partial t}=\alpha\,\frac{\partial^2 T}{\partial x^2},
\qquad \alpha=\frac{k}{\rho c_p}.
\]

\medskip

A prescribed (time-dependent) heat flux at the left boundary:
\[
q(0,t) \equiv -k\left.\frac{\partial T}{\partial x}\right|_{x=0} = g(t).
\]

\medskip
\begin{itemize}
\item $g(t)>0$ means \emph{heat enters} the domain at $x=0$ (inward flux).
\item $g(t)=0$ corresponds to an insulated boundary.
\end{itemize}

\end{frame}


\begin{frame}{Finite Difference Grid and Notation}
Uniform grid:
\[
x_j=j\,\Delta x,\qquad j=0,1,\dots,N,\qquad \Delta x=\frac{L}{N}.
\]

\medskip

Discrete temperatures:
\[
T_j^n \approx T(x_j,t^n),\qquad t^n=n\,\Delta t.
\]

\medskip

Define the diffusion number:
\[
\lambda=\frac{\alpha\,\Delta t}{\Delta x^2}.
\]

\medskip

Interior FTCS update ($j=1,\dots,N-1$):
\[
T_j^{n+1}=T_j^n+\lambda\left(T_{j+1}^n-2T_j^n+T_{j-1}^n\right).
\]

\end{frame}


\begin{frame}{Imposing a Prescribed Flux at $x=0$: From Flux to Gradient}
The boundary condition is specified as a \emph{flux}:
\[
-k\left.\frac{\partial T}{\partial x}\right|_{x=0}=g(t).
\]

\medskip

Convert to a prescribed gradient:
\[
\left.\frac{\partial T}{\partial x}\right|_{x=0}=-\frac{g(t)}{k}.
\]

\medskip

At time level $n$, define
\[
\gamma^n \equiv \left.\frac{\partial T}{\partial x}\right|_{x=0,t=t^n}
= -\frac{g(t^n)}{k}.
\]

\end{frame}


\begin{frame}{Taylor-Series Boundary Update at $j=0$ (Second-Order Accurate)}
Taylor expansion from node $0$ to node $1$:
\[
T_1^n
=
T_0^n
+
\Delta x\,\gamma^n
+
\frac{\Delta x^2}{2}\left.T_{xx}\right|_{0}^n
+\mathcal{O}(\Delta x^3).
\]

\medskip

Solve for the boundary curvature:
\[
\left.T_{xx}\right|_{0}^n
\approx
\frac{2}{\Delta x^2}\left(T_1^n-T_0^n-\Delta x\,\gamma^n\right).
\]

\medskip

Use the heat equation at $x=0$:
\[
\frac{dT_0}{dt}=\alpha \left.T_{xx}\right|_{0}.
\]

Forward Euler in time gives the boundary node update:
\[
\boxed{
T_0^{n+1}
=
T_0^n
+
2\lambda\left(T_1^n-T_0^n-\Delta x\,\gamma^n\right)
}
\qquad
\left(\gamma^n=-\frac{g(t^n)}{k}\right).
\]

\end{frame}



\begin{frame}{Algorithm Summary (FTCS with Prescribed Flux at $x=0$)}
At each time step $n\to n+1$:

\begin{enumerate}
\item Evaluate the prescribed flux $g(t^n)$ and compute
\[
\gamma^n = -\frac{g(t^n)}{k}.
\]
\item Update the left boundary node ($j=0$):
\[
T_0^{n+1}
=
T_0^n
+
2\lambda\left(T_1^n-T_0^n-\Delta x\,\gamma^n\right).
\]
\item Update interior nodes ($j=1,\dots,N-1$):
\[
T_j^{n+1}=T_j^n+\lambda\left(T_{j+1}^n-2T_j^n+T_{j-1}^n\right).
\]
\item Enforce the right boundary condition at $x=L$ (Dirichlet or Robin), as given.
\end{enumerate}

\end{frame}
% -------------------------

% -------------------------

% -------------------------

% -------------------------

% -------------------------

% -------------------------

% -------------------------
\begin{frame}{Boundary Conditions for the Cooling Sphere HW3}

We solve the radially symmetric heat equation in a sphere:
\[
\frac{\partial T}{\partial t}
=
\alpha\left[\frac{1}{r^2}\frac{\partial}{\partial r}\left(r^2\frac{\partial T}{\partial r}\right)\right],
\qquad 0<r<R.
\]

\medskip

Two boundary conditions are required:
\begin{itemize}
  \item \textbf{Center symmetry (Neumann):}
  \[
  \left.\frac{\partial T}{\partial r}\right|_{r=0}=0
  \quad\Longleftrightarrow\quad
  q_r(0,t)=0.
  \]
  \item \textbf{Surface convection (Robin):}
  \[
  -k\left.\frac{\partial T}{\partial r}\right|_{r=R}
  =
  h\left(T(R,t)-T_\infty\right).
  \]
\end{itemize}

\end{frame}
% -------------------------
\begin{frame}{Neumann BC at $r=0$ via a Balance Law (Center Control Volume)}

Consider a small spherical control volume around the origin,
with radius $\Delta r/2$.

\[
\rho c_p V\,\frac{\mathrm{d}T_0}{\mathrm{d}t} = -qA,
\]
where $q$ is the outward heat flux at $r=\Delta r/2$.

\medskip

Geometry:
\[
V=\frac{4}{3}\pi\left(\frac{\Delta r}{2}\right)^3,
\qquad
A=4\pi\left(\frac{\Delta r}{2}\right)^2.
\]

Approximate the conductive flux across the boundary:
\[
q \approx -k\,\frac{T_1-T_0}{\Delta r}.
\]

\end{frame}

\begin{frame}{Neuman BC at $r=0$: Conduction Balances Convection}
Substitute and simplify:
\[
\frac{\mathrm{d}T_0}{\mathrm{d}t}
=
6\alpha\,\frac{T_1-T_0}{\Delta r^2}.
\]

Forward Euler in time gives the \textbf{center update}:
\[
T_0^{n+1} = T_0^n + 6\lambda\left(T_1^n - T_0^n\right),
\qquad
\lambda=\frac{\alpha\Delta t}{\Delta r^2}.
\]

\end{frame}
% -------------------------
\begin{frame}{Robin BC at $r=R$: Conduction Balances Convection}

At the surface, the boundary condition is
\[
-k\left.\frac{\partial T}{\partial r}\right|_{r=R}
=
h\left(T(R,t)-T_\infty\right).
\]

\medskip

Interpretation:
\begin{itemize}
  \item $-k\,\partial T/\partial r$ is the \textbf{conductive heat flux leaving the solid}
  \item $h(T-T_\infty)$ is the \textbf{convective heat flux into the air}
\end{itemize}

\medskip

This is a \textbf{Robin (mixed) boundary condition} because it couples:
\[
\text{temperature at the boundary} \quad \text{and} \quad \text{temperature gradient at the boundary}.
\]

\medskip

In a finite difference method, we enforce this condition at each time step
to determine the surface behavior.

\end{frame}
% -------------------------
\begin{frame}{Robin BC Implementation: Ghost Node at the Surface}

Approximate the surface derivative using a centered difference:
\[
\left.\frac{\partial T}{\partial r}\right|_{r=R}
\approx
\frac{T_{N+1}^n - T_{N-1}^n}{2\Delta r}.
\]

Insert into the Robin condition:
\[
-k\,\frac{T_{N+1}^n - T_{N-1}^n}{2\Delta r}
=
h\left(T_N^n - T_\infty\right).
\]

Solve for the ghost value $T_{N+1}^n$:
\[
\boxed{
T_{N+1}^n
=
T_{N-1}^n
-
2\Delta r\,\frac{h}{k}\left(T_N^n - T_\infty\right).
}
\]

\medskip

Then the surface node update uses the same stencil as interior nodes,
with $T_{N+1}^n$ substituted where $T_{N+1}$ appears.

\end{frame}
% -------------------------
\begin{frame}{Summary: How We Enforce Boundary Conditions}

\begin{itemize}
  \item \textbf{Center ($r=0$), Neumann / symmetry:}
  \[
  T_0^{n+1} = T_0^n + 6\lambda\left(T_1^n - T_0^n\right).
  \]
  \item \textbf{Surface ($r=R$), Robin / convection:}
  \[
  -k\left.\frac{\partial T}{\partial r}\right|_{r=R}
  =
  h\left(T_N - T_\infty\right),
  \qquad
  T_{N+1}^n
=
T_{N-1}^n
-
2\Delta r\,\frac{h}{k}\left(T_N^n - T_\infty\right).
  \]
  \item \textbf{Interior ($i=1,\dots,N-1$):} use the standard spherical FTCS stencil.
\end{itemize}

\medskip
\textbf{Takeaway:} boundary conditions are enforced \emph{at every time step}, and they
determine the updates at nodes where the standard interior stencil does not apply.

\end{frame}
% -------------------------

% -------------------------

% -------------------------

% -------------------------


\end{document}
