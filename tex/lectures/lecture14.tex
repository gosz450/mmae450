\documentclass[aspectratio=169]{beamer}

\usepackage{amsmath,amssymb}
\usepackage{bm}
\usepackage{physics}
\usepackage{tikz}
\usepackage{pgfplots}
\pgfplotsset{compat=1.18}

\title{Balance Laws}
\subtitle{Important Quantities with Units}
%\author{Mike Gosz}
\date{}

\begin{document}

% ------------------------------------------------------------
\begin{frame}
  
%========================================================
% Table 1: Fields + fluxes + sources in balance laws
%========================================================
\begin{table}[h!]
\centering
\renewcommand{\arraystretch}{1.25}
\begin{tabular}{lll}
\hline
\textbf{Symbol} & \textbf{Description} & \textbf{SI Units} \\
\hline

$\rho$ & Mass density & $\mathrm{kg/m^3}$ \\
$\mathbf{v}$ & Velocity vector & $\mathrm{m/s}$ \\
$\rho \mathbf{v}$ & Momentum per unit volume & $\mathrm{kg/(m^2\,s)}$ \\

$\boldsymbol{\sigma}$ & Cauchy stress tensor & $N/m^2$ \\
$\boldsymbol{\sigma}\mathbf{n}$ & Traction vector on a surface & $\mathrm{N/m^2}$ \\

$\mathbf{b}$ & Body force per unit mass (e.g., gravity) & $\mathrm{m/s^2}$ \\
$\rho \mathbf{b}$ & Body force per unit volume & $\mathrm{N/m^3}$ \\

$\mathbf{q}$ & Heat flux vector & $\mathrm{W/m^2}$ \\
$S$ & Volumetric heat source (heat generation rate) & $\mathrm{W/m^3}$ \\

\hline
\end{tabular}
\caption{Common fields, fluxes, and sources appearing in mass, momentum, and energy balance laws.}
\label{tab:balance_symbols_fluxes_sources}
\end{table}

\end{frame}

\begin{frame}
%========================================================
% Table 2: Energy variables + constitutive/thermal properties
%========================================================
\begin{table}[h!]
\centering
\renewcommand{\arraystretch}{1.25}
\begin{tabular}{lll}
\hline
\textbf{Symbol} & \textbf{Description} & \textbf{SI Units} \\
\hline

$T$ & Temperature & $\mathrm{K}$ \\
$e$ & Internal energy per unit mass & $\mathrm{J/kg}$ \\
$E$ & Total energy per unit mass $\left(E=e+\frac{1}{2}\mathbf{v}\cdot\mathbf{v}\right)$ & $\mathrm{J/kg}$ \\
$\rho e$ & Internal energy per unit volume & $\mathrm{J/m^3}$ \\
$\rho E$ & Total energy per unit volume & $\mathrm{J/m^3}$ \\

$k$ & Thermal conductivity & $\mathrm{W/(m\,K)}$ \\
$c_p$ & Specific heat at constant pressure & $\mathrm{J/(kg\,K)}$ \\
$c_v$ & Specific heat at constant volume & $\mathrm{J/(kg\,K)}$ \\

\hline
\end{tabular}
\caption{Energy variables and common thermal material properties used in the energy equation.}
\label{tab:energy_symbols_properties}
\end{table}


\end{frame}


\begin{frame}

%========================================================
% Table 3: Operators and Derived Quantities
%========================================================
\begin{table}[h!]
\centering
\renewcommand{\arraystretch}{1.25}
\begin{tabular}{lll}
\hline
\textbf{Symbol} & \textbf{Description} & \textbf{SI Units} \\
\hline

$\nabla$ & Gradient operator & $\mathrm{1/m}$ \\

$\nabla \cdot \mathbf{v}$ & Velocity divergence & $\mathrm{1/s}$ \\

$\nabla \cdot (\rho \mathbf{v})$ & Mass flux divergence & $\mathrm{kg/(m^3\,s)}$ \\

$\nabla \cdot \boldsymbol{\sigma}$ & Stress divergence (force density) & $\mathrm{N/m^3}$ \\

$\nabla \cdot \mathbf{q}$ & Heat flux divergence & $\mathrm{W/m^3}$ \\

$\rho \mathbf{v} \otimes \mathbf{v}$ & Momentum flux tensor & $\mathrm{kg/(m\,s^2)}$ \\

$\boldsymbol{\sigma}\mathbf{v}$ & Mechanical power flux & $\mathrm{W/m^2}$ \\

\hline
\end{tabular}
\caption{Differential operators and commonly appearing derived quantities in balance laws.}
\label{tab:operators_derived_balance}
\end{table}

\end{frame}

\end{document}
