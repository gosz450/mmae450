
\documentclass[aspectratio=169]{beamer}

% --- Theme / packages ---
% -------------------------------------------------
% Packages
% -------------------------------------------------
\usepackage{amsmath, amsfonts}
\usepackage{graphicx}
\usepackage{booktabs}
\usepackage{hyperref}
\usepackage{bm}
\usepackage{siunitx}
\usepackage{listings}

% Figure search paths (relative to tex/lectures/)
\graphicspath{{../figures/lectures/}{../figures/shared/}}

% -------------------------------------------------
% Notes (speaker notes)
% -------------------------------------------------
\usepackage{pgfpages}
% Uncomment ONE of these for speaker notes:
% \setbeameroption{show notes} % notes only (for printing notes)
% \setbeameroption{show notes on second screen=right} % slides + notes

% -------------------------------------------------
% TikZ
% -------------------------------------------------
\usepackage{tikz}
\usetikzlibrary{matrix, calc}
\usepackage{xcolor} % (tikz loads xcolor, but explicit is fine)
\usetikzlibrary{arrows.meta, decorations.pathreplacing}


% -------------------------------------------------
% Themes
%--------------------------------------------------
\usetheme{default}
\usecolortheme{default}
\setbeamertemplate{navigation symbols}{}

% -----------------------------
% Code formatting
% -----------------------------
\definecolor{codegray}{RGB}{245,245,245}

\lstset{
  backgroundcolor=\color{codegray},
  basicstyle=\ttfamily\small,
  frame=single,
  breaklines=true,
  showstringspaces=false
}

% --- Shortcuts ---
%\newcommand{\dd}{\,\mathrm{d}}
\newcommand{\Grad}{\nabla}
\newcommand{\Div}{\nabla\!\cdot}
\newcommand{\bx}{\bm{x}}
\newcommand{\bn}{\bm{n}}
\newcommand{\bq}{\bm{q}}

%==========================
% Flipbook macro (put in preamble or before frames)
%==========================
\newcommand{\GSFlipFrame}[2]{%
\begin{frame}{Gauss--Seidel sweep (current point $(#1,#2)$)}
\centering
\hspace{3.5cm}
\vspace{3cm}
\begin{tikzpicture}[scale=2.2, every node/.style={font=\small}]
  % -----------------------------
  % USER SETTINGS
  % -----------------------------
  \def\Nx{6}   % i = 0..Nx-1
  \def\Ny{4}   % j = 0..Ny-1

  % current stencil location (interior)
  \def\ci{#1}
  \def\cj{#2}

  % Point spacing
  \def\s{0.85}

  % -----------------------------
  % STYLES
  % -----------------------------
  \tikzset{
    gpG/.style={circle, fill=green!70!black, inner sep=1.1pt},
    gpR/.style={circle, fill=red!75!black,   inner sep=1.1pt},
    gpbG/.style={circle, fill=green!70!black, inner sep=1.5pt},
    gpbR/.style={circle, fill=red!75!black,   inner sep=1.5pt},
    gpc/.style={circle, fill=blue, inner sep=1.7pt},
    stline/.style={dashed, thick},
    labij/.style={font=\normalsize},
    labp/.style={font=\scriptsize, text=gray!70},
  }

  % -----------------------------
  % OUTER RECTANGLE (domain)
  % -----------------------------
  \draw[thick] (0,0) rectangle ({(\Nx-1)*\s},{(\Ny-1)*\s});

  % -----------------------------
  % GRID POINTS + LABELS
  % -----------------------------
  \foreach \j in {0,...,\numexpr\Ny-1\relax} {
    \foreach \i in {0,...,\numexpr\Nx-1\relax} {

      \pgfmathsetmacro{\x}{\i*\s}
      \pgfmathsetmacro{\y}{\j*\s}

      % global index p = i + j*Nx (as in your first figure)
      \pgfmathtruncatemacro{\p}{\i + \j*\Nx}

      % boundary?
      \pgfmathtruncatemacro{\isB}{ (\i==0) || (\i==\Nx-1) || (\j==0) || (\j==\Ny-1) }
      % current center?
      \pgfmathtruncatemacro{\isC}{ (\i==\ci) && (\j==\cj) }

      % "already updated" (GS ordering): j<cj OR (j==cj and i<ci)
      \pgfmathtruncatemacro{\isUpdated}{ (\j<\cj) || ((\j==\cj) && (\i<\ci)) }

      % FORCE left boundary (i==0) and bottom boundary (j==0) to be green
      \pgfmathtruncatemacro{\forceGreen}{ \isB }

      % decide color state:
      % - center: blue
      % - else green if forced green OR updated
      % - else red
      \ifnum\isC=1
        \node[gpc] (P\i\j) at (\x,\y) {};
      \else
        \pgfmathtruncatemacro{\isGreen}{ (\forceGreen==1) || (\isUpdated==1) }
        \ifnum\isB=1
          \ifnum\isGreen=1
            \node[gpbG] (P\i\j) at (\x,\y) {};
          \else
            \node[gpbR] (P\i\j) at (\x,\y) {};
          \fi
        \else
          \ifnum\isGreen=1
            \node[gpG] (P\i\j) at (\x,\y) {};
          \else
            \node[gpR] (P\i\j) at (\x,\y) {};
          \fi
        \fi
      \fi

      % Labels (i,j) above-right; p below-right
      \node[labij, anchor=south west] at (\x+0.05,\y+0.05) {$(\i,\j)$};
      \node[labp,  anchor=north west] at (\x+0.05,\y-0.05) {$\p$};
    }
  }

  % -----------------------------
  % 5-POINT STENCIL (dashed)
  % -----------------------------
  \pgfmathsetmacro{\xc}{\ci*\s}
  \pgfmathsetmacro{\yc}{\cj*\s}

  \draw[stline] (\xc,\yc) -- ({(\ci+1)*\s},\yc);
  \draw[stline] (\xc,\yc) -- ({(\ci-1)*\s},\yc);
  \draw[stline] (\xc,\yc) -- (\xc,{(\cj+1)*\s});
  \draw[stline] (\xc,\yc) -- (\xc,{(\cj-1)*\s});

  % Mapping note
  %\node[font=\scriptsize, align=left] at ({(1)*\s+1.65},{(0)*\s-0.55})
  %{$p = i + j\,N_x$};
\end{tikzpicture}
\end{frame}%
}











\title{1D Wave Equation}
\subtitle{Derivation, Finite Differences, and Stability}
\author{MMAE 450}
\date{}

\begin{document}


\tikzset{
  vectorArrow/.style={
    thick,
    ->,
    >=Latex
  }
 }
 
 \tikzset{
  axisArrow/.style={
    thin,
    ->,
    >=Stealth
  }
}

% ==========================================================
\begin{frame}
  \titlepage
\end{frame}

% ==========================================================
\begin{frame}{Physical model: axial waves in a slender bar}
\centering
\begin{tikzpicture}[scale=0.85]
  % Geometry
  \def\L{10.0}
  \def\H{0.6}
  \def\clampPad{0.4}

  % Bar
  \draw[fill=gray!5, axisLine] (0,-\H/2) rectangle (\L,\H/2);

  % Clamp at x=0
  \path[fill=gray!55] (0,-\H/2-\clampPad) rectangle (0,\H/2+\clampPad);
  \draw[axisLine]     (0,-\H/2-\clampPad) rectangle (0,\H/2+\clampPad);
  \foreach \y in {-0.5,-0.3,...,0.5}{
    \draw[axisLine] (0,\y) -- (-0.35,\y-0.25);
  }
  \node[figureLabel, below=2pt, inner sep=1pt] at (0,-\H/2-\clampPad) {$x=0$ (fixed)};

  % Traction at x=L (compressive to the left)
  \draw[vectorArrow] (\L+0.85,0) -- (\L+0.05,0);
  \node[figureLabel, above, inner sep=1pt] at (\L+0.45,0.22) {$\sigma_0$};
  \node[figureLabel, below=2pt, inner sep=1pt] at (\L,-\H/2-\clampPad) {$x=L$};

  % x-axis
  \draw[axisArrow] (0,-1.6) -- (\L+0.8,-1.6);
  \node[axisLabel, right, inner sep=1pt] at (\L+0.8,-1.6) {$x$};

  % Displacement label
  \node[figureLabel] at (5,1.0) {axial displacement $u(x,t)$};
\end{tikzpicture}

\vspace{0.2cm}
\begin{block}{Assumptions}
\begin{itemize}
  \item No body force; constant area $A$ and density $\rho$
  \item Linear elasticity: $\sigma = E\,\varepsilon$, with $\varepsilon = \dfrac{\mathrm{d}u}{\mathrm{d}x}$
  \item Small displacement gradients: $|\dfrac{\mathrm{d}u}{\mathrm{d}x}| \ll 1$ (geometric linearization)
\end{itemize}
\end{block}
\end{frame}

% ==========================================================
\begin{frame}
\centering
{\Large Derivation of the 1D wave equation}

\vspace{1em}

\begin{tikzpicture}[x=1cm,y=1cm,scale=1.0]
  \def\dx{3.0}
  \def\h{0.8}

  % differential slice
  \draw[thick, fill=gray!20] (0,0) rectangle (\dx,\h);
  
  % face locations
  \draw[thin] (0,0) -- (0,-0.45);
  \draw[thin] (\dx,0) -- (\dx,-0.45);
  \node[below] at (0,-0.45) {$x$};
  \node[below] at (\dx+.2,-0.45) {$x+\Delta x$};

  % x-axis
  \draw[axisArrow] (-0.3,-1.05) -- (\dx+0.8,-1.05)
    node[right] {$x$};

  % forces on faces
  \draw[vectorArrow] (0,0.5*\h) -- (-1.0,0.5*\h)
    node[midway,above, xshift=-4mm] {$A\,\sigma(x,t)$};

  \draw[vectorArrow] (\dx,0.5*\h) -- (\dx+1.0,0.5*\h)
        node[midway,above,xshift=8mm] {$A\,\sigma(x+\Delta x,t)$};

  % dx label
  \draw[<->,thin] (0,-0.25) -- (\dx,-0.25);
  \node[below] at (0.5*\dx,-0.25) {$\Delta x$};

\end{tikzpicture}

\vspace{0.75em}

\begin{itemize}
  \item Axial stress acts on each face of the slice
  \item Force on a face is $A\,\sigma(x,t)$
  \item Mass of the slice is $\rho A\,\Delta x$
\end{itemize}
\end{frame}
% ==========================================================
\begin{frame}
\centering
{\Large Derivation of the 1D wave equation}

\vspace{1em}

\begin{block}{Linear momentum balance (Newton's 2nd law)}
\[
A\,\sigma(x+\Delta x,t)-A\,\sigma(x,t)
=
\rho A\,\Delta x\,\frac{\partial^2u}{\partial t^2}
\]

Divide through by $A\Delta x$ and take limit as $\Delta x \rightarrow 0$.

\[
\frac{\partial\sigma}{\partial x} = \rho \frac{\partial^2u}{\partial x^2}
\]
\end{block}


\begin{block}{Constitutive law + kinematics (small strain)}
\[
\varepsilon = \frac{\partial u}{\partial x},
\qquad
\sigma = E\,\varepsilon = E\,\frac{\partial u}{\partial x}
\]

\[
\frac{\partial \sigma}{\partial x}
=
E\,\frac{\partial^2u}{\partial x^2}
\]
\end{block}
\end{frame}


\begin{frame}
\begin{block}{1D wave equation}
\[
\frac{\partial^2u}{\partial t^2}
=
c^2\,\frac{\partial^2u}{\partial x^2},
\qquad
c=\sqrt{\frac{E}{\rho}}
\]

The parameter $c$ is the \emph{wave speed}
\end{block}
\end{frame}







%===============================================
\begin{frame}{Finite differences: grid and discrete operators}

Uniform mesh:
\[
x_j = j\,\Delta x,
\qquad
t^n = n\,\Delta t,
\qquad
u_j^n \approx u(x_j,t^n)
\]

\begin{block}{Second derivatives (centered)}
\[
\frac{\partial^2 u}{\partial t^2}(x_j,t^n)
\;\approx\;
\frac{u_j^{\,n+1}-2u_j^{\,n}+u_j^{\,n-1}}{\Delta t^2}
\]

\[
\frac{\partial^2 u}{\partial x^2}(x_j,t^n)
\;\approx\;
\frac{u_{j+1}^{\,n}-2u_j^{\,n}+u_{j-1}^{\,n}}{\Delta x^2}
\]
\end{block}

Define the Courant number (wave CFL):
\[
r \equiv \frac{c\,\Delta t}{\Delta x}.
\]

\end{frame}

% ==========================================================
\begin{frame}{Finite differences: explicit update (central in time and space)}
Start from $u_{tt}=c^2u_{xx}$ at $(x_j,t^n)$ and substitute the centered formulas:

\[
\frac{u_j^{n+1}-2u_j^{n}+u_j^{n-1}}{\Delta t^2}
=
c^2
\frac{u_{j+1}^{n}-2u_{j}^{n}+u_{j-1}^{n}}{\Delta x^2}
\]

\begin{block}{Explicit time-marching scheme (CTCS)}
\[
\boxed{
u_j^{n+1}
=
2u_j^{n}-u_j^{n-1}
+
r^2\left(u_{j+1}^{n}-2u_{j}^{n}+u_{j-1}^{n}\right)
}
\]
\end{block}


\end{frame}

% ==========================================================

\begin{frame}{Boundary conditions: displacement and stress}

\begin{block}{Dirichlet (prescribed displacement)}
At a boundary where the displacement is specified, e.g. at $x=0$:
\[
u(0,t) = u^*(t)
\]

On the grid, this is enforced \emph{strongly} at every time step:
\[
u_0^{\,n} = u^*(t^n)
\]

\begin{itemize}
  \item No update equation is used at this node
  \item The boundary value is imposed directly
  \item Interior updates use this known value
\end{itemize}
\end{block}

\end{frame}
%==================================================
\begin{frame}{Neumann boundary condition via Taylor expansion}

\begin{block}{Taylor expansion about the boundary $x=0$}
At a fixed time $t$, expand $u(x,t)$ about $x=0$:
\[
u(\Delta x,t)
=
u(0,t)
+
\Delta x\,\frac{\partial u}{\partial x}(0,t)
+
\frac{\Delta x^2}{2}\,
\frac{\partial^2 u}{\partial x^2}(0,t)
+
\mathcal{O}(\Delta x^3)
\]

Impose the Neumann condition (zero stress / zero slope):
\[
\frac{\partial u}{\partial x}(0,t)=0
\]

Solve for the second derivative:
\[
\frac{\partial^2 u}{\partial x^2}(0,t)
\;\approx\;
\frac{2\big(u(\Delta x,t)-u(0,t)\big)}{\Delta x^2}
\]
\end{block}

\end{frame}
%=========================================
\begin{frame}
\begin{block}{Boundary update equation}
On the grid:
\[
\frac{\partial^2 u}{\partial x^2}(0,t^n)
\;\approx\;
\frac{2\big(u_1^{\,n}-u_0^{\,n}\big)}{\Delta x^2}
\]

Use the centered time discretization:
\[
\frac{\partial^2 u}{\partial t^2}(0,t^n)
\;\approx\;
\frac{u_0^{\,n+1}-2u_0^{\,n}+u_0^{\,n-1}}{\Delta t^2}
\]

Substitute into $u_{tt}=c^2u_{xx}$ to obtain:
\[
\boxed{
u_0^{\,n+1}
=
2u_0^{\,n}-u_0^{\,n-1}
+
2r^2\big(u_1^{\,n}-u_0^{\,n}\big)
}
\]
\[
r=\frac{c\,\Delta t}{\Delta x}
\]
\end{block}

\end{frame}

%===============================================

\begin{frame}{Von Neumann stability: Fourier mode substitution}

Assume a Fourier mode on an infinite / periodic grid:
\[
u_j^n = A^n e^{i\theta j},
\qquad \theta\in[0,\pi].
\]

Substitute into the explicit update:
\[
A^{n+1}e^{i\theta j}
=
2A^{n}e^{i\theta j}-A^{n-1}e^{i\theta j}
+
r^2A^{n}\!\left(e^{i\theta (j+1)}-2e^{i\theta j}+e^{i\theta (j-1)}\right).
\]

\begin{block}{Divide through and simplify}
Divide by $A^{n-1}e^{i\theta j}$:
\[
\frac{A^{n+1}}{A^{n-1}}
=
2\frac{A^{n}}{A^{n-1}}
-
1
+
r^2\frac{A^{n}}{A^{n-1}}
\left(e^{i\theta}-2+e^{-i\theta}\right).
\]
\end{block}
\end{frame}
%----------------------------------------------
\begin{frame}

\begin{block}

Use
\[
e^{i\theta}-2+e^{-i\theta}=-4\sin^2\!\frac{\theta}{2},
\]
to obtain
\[
\frac{A^{n+1}}{A^{n-1}}
-
2\!\left(1-2r^2\sin^2\frac{\theta}{2}\right)
\frac{A^{n}}{A^{n-1}}
+
1
=
0.
\]
\end{block}

\begin{block}{Amplification factor}
Define
\[
G \;\equiv\; \frac{A^{n}}{A^{n-1}}.
\]
Then the quadratic becomes
\[
G^2
-
2\!\left(1-2r^2\sin^2\frac{\theta}{2}\right)G
+
1
=
0.
\]
\end{block}
\end{frame}




%===============================================
\begin{frame}{Roots of the characteristic equation}

Define
\[
a(\theta)\;\equiv\;1-2r^2\sin^2\!\frac{\theta}{2}.
\]

Then the amplification factor satisfies
\[
G^2 - 2a(\theta)\,G + 1 = 0,
\]
with roots
\[
G_{1,2} = a \pm \sqrt{a^2-1}.
\]

\begin{block}{Key fact}
The constant term is $+1$, so
\[
G_1G_2 = 1.
\]
\end{block}

\end{frame}
%===============================================
\begin{frame}{Stable case: complex roots (bounded oscillations)}

\begin{block}{When do we get complex roots?}
If
\[
a^2-1 \le 0
\qquad\Longleftrightarrow\qquad
|a|\le 1,
\]
then the roots are complex conjugates:
\[
G_{1,2} = a \pm i\sqrt{1-a^2}.
\]
\end{block}

\begin{block}{Magnitude of the roots}
\[
|G_{1,2}|^2 = a^2 + (1-a^2)=1
\qquad\Longrightarrow\qquad
|G_{1}|=|G_{2}|=1.
\]
So each Fourier mode is bounded in time (no growth).
\end{block}

\end{frame}
%===============================================
\begin{frame}{Unstable case: real roots (blow-up)}

\begin{alertblock}{When are the roots real?}
If
\[
a^2-1>0,
\]
then $G_{1,2}$ are real.
\end{alertblock}

\begin{block}{Why that implies blow-up}
Because
\[
G_1G_2=1,
\]
it is impossible for both real roots to satisfy $|G|\le 1$
unless $|G_1|=|G_2|=1$ (which occurs only at the boundary case $a=\pm1$).

Therefore, when $a^2-1>0$ one root must satisfy
\[
|G|>1,
\]
and that Fourier component grows exponentially with $n$.
\end{block}

\end{frame}
%===============================================
\begin{frame}{CFL stability condition for the wave equation}

We require stability for \emph{all} Fourier modes $\theta\in[0,\pi]$:
\[
|a(\theta)| \le 1,
\qquad
a(\theta)=1-2r^2\sin^2\!\frac{\theta}{2}.
\]

\begin{block}{Upper bound is automatic}
Since $\sin^2(\theta/2)\ge 0$,
\[
a(\theta)\le 1 \quad\text{always.}
\]
\end{block}

\begin{block}{Lower bound gives the restriction}
The smallest value occurs at $\sin^2(\theta/2)=1$ (i.e., $\theta=\pi$):
\[
a_{\min}=1-2r^2.
\]
Require $a_{\min}\ge -1$:
\[
1-2r^2 \ge -1
\quad\Longrightarrow\quad
r^2\le 1
\quad\Longrightarrow\quad
\boxed{\,r=\frac{c\,\Delta t}{\Delta x}\le 1\,}.
\]
\end{block}

\end{frame}
%===============================================

%===============================================


% ==========================================================

% ==========================================================

\end{document}
