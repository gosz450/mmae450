\documentclass[aspectratio=169]{beamer}

% -------------------------------------------------
% MMAE 450 – Lecture: Indicial Notation & Balance Laws
% Date: 2/10/2026
% Starting draft (TikZ figures can be inserted later)
% -------------------------------------------------

\usepackage{amsmath,amssymb,bm}
\usepackage{physics}
\usepackage{mathtools}
\usepackage{graphicx}
\usepackage{pgfplots}
\pgfplotsset{compat=1.18}


\graphicspath{{../figures/lectures/}}


\title[MMAE 450]{Indicial Notation, Material Derivative, and Balance Laws}
%\author{M. R. Gosz}
\date{}

% -------------------------------------------------
\begin{document}

% -------------------------------------------------
\begin{frame}
  \titlepage
\end{frame}

% -------------------------------------------------
\begin{frame}{Outline (if time allows)}
\begin{itemize}
  \item Why indicial notation?
  \item Common operators in component form
  \item Material time derivative
  \item Balance laws: mass, momentum, energy
  \item Advection--diffusion equation
  \item Reduction to 1D
\end{itemize}
\end{frame}

% =================================================
\section{Indicial Notation}
% =================================================

% -------------------------------------------------
\begin{frame}{Why Indicial Notation?}
\begin{itemize}
  \item Compact representation of vector/tensor equations
  \item Eliminates ambiguity in differentiation
  \item Essential for deriving and implementing balance laws
  \item Natural bridge to component-based numerical methods
\end{itemize}
\end{frame}

% -------------------------------------------------
\begin{frame}{Basic Conventions}
\begin{block}{Index notation rules}
\begin{itemize}
  \item Indices: $i,j,k = 1,2,3$
  \item Repeated indices imply summation (Einstein convention)
  \item Free indices must match on both sides of an equation
\end{itemize}
\end{block}

\begin{block}{Examples}
\[
\mathbf{v} = v_i\,\mathbf{e}_i,
\qquad
\mathbf{a}\cdot\mathbf{b} = a_i b_i
\]
\end{block}
\end{frame}

% -------------------------------------------------
\begin{frame}{Vectors in a Cartesian Basis}
Write vectors in terms of components and Cartesian base vectors:
\[
\mathbf{a} = a_i\,\mathbf{e}_i,
\qquad
\mathbf{b} = b_j\,\mathbf{e}_j
\]
\begin{itemize}
  \item $\mathbf{e}_i$ are orthonormal Cartesian base vectors
  \item Components $a_i, b_j$ are scalars
\end{itemize}
\end{frame}

% -------------------------------------------------
\begin{frame}{Dot Product in Index Form}
Start from the basis expansion:
\[
\mathbf{a}\cdot\mathbf{b}
= (a_i\mathbf{e}_i)\cdot(b_j\mathbf{e}_j)
= a_i b_j\,(\mathbf{e}_i\cdot\mathbf{e}_j)
\]
For Cartesian base vectors,
\[
\mathbf{e}_i\cdot\mathbf{e}_j = \delta_{ij}
\]
so
\[
\mathbf{a}\cdot\mathbf{b}
= a_i b_j \delta_{ij}
= a_i b_i.
\]
\end{frame}

% -------------------------------------------------
\begin{frame}{Second-Order Tensors in a Cartesian Basis}
Represent a second-order tensor using dyads:
\[
\boldsymbol{\sigma} = \sigma_{ij}\,(\mathbf{e}_i\otimes\mathbf{e}_j)
\]
\begin{itemize}
  \item $\sigma_{ij}$ are the tensor components
  \item $(\mathbf{e}_i\otimes\mathbf{e}_j)\mathbf{v} = \mathbf{e}_i(\mathbf{e}_j\cdot\mathbf{v})$
\end{itemize}
\end{frame}

% -------------------------------------------------
\begin{frame}{Traction: $\boldsymbol{\sigma}\cdot\mathbf{n}$ in Components}
Let
\[
\mathbf{n} = n_k\,\mathbf{e}_k,
\qquad
\boldsymbol{\sigma} = \sigma_{ij}(\mathbf{e}_i\otimes\mathbf{e}_j).
\]
Then the traction vector is
\[
\boldsymbol{\sigma}\cdot\mathbf{n}
= \sigma_{ij}(\mathbf{e}_i\otimes\mathbf{e}_j)\cdot(n_k\mathbf{e}_k)
= \sigma_{ij}n_k\,\mathbf{e}_i(\mathbf{e}_j\cdot\mathbf{e}_k).
\]
Using $\mathbf{e}_j\cdot\mathbf{e}_k=\delta_{jk}$,
\[
\boldsymbol{\sigma}\cdot\mathbf{n}
= \sigma_{ij}n_k\,\delta_{jk}\,\mathbf{e}_i
= (\sigma_{ij}n_j)\,\mathbf{e}_i.
\]
So, in components:
\[
(\boldsymbol{\sigma}\cdot\mathbf{n})_i = \sigma_{ij}n_j.
\]
\end{frame}


% =================================================
\section{Operators in Component Form}
% =================================================

% -------------------------------------------------
\begin{frame}{$\nabla\cdot\boldsymbol{\sigma}$ in a Cartesian Basis}
Write the gradient operator and stress tensor in a Cartesian basis:
\[
\nabla = \mathbf{e}_j\,\frac{\partial}{\partial x_j},
\qquad
\boldsymbol{\sigma} = \sigma_{ij}\,(\mathbf{e}_i\otimes\mathbf{e}_j).
\]
Then
\[
\nabla\cdot\boldsymbol{\sigma}
=
\left(\mathbf{e}_j\frac{\partial}{\partial x_j}\right)\cdot
\left(\sigma_{ik}\,\mathbf{e}_i\otimes\mathbf{e}_k\right).
\]
Use the identity $\mathbf{e}_j\cdot(\mathbf{e}_i\otimes\mathbf{e}_k)=\delta_{ji}\,\mathbf{e}_k$ to obtain
\[
\nabla\cdot\boldsymbol{\sigma}
=
\frac{\partial \sigma_{ik}}{\partial x_j}\,\delta_{ji}\,\mathbf{e}_k
=
\left(\frac{\partial \sigma_{ij}}{\partial x_j}\right)\mathbf{e}_i.
\]
So, in components:
\[
(\nabla\cdot\boldsymbol{\sigma})_i = \frac{\partial \sigma_{ij}}{\partial x_j}.
\]
\end{frame}


% -------------------------------------------------
\begin{frame}{$\boldsymbol{\sigma}\cdot\mathbf{n}$}
\[
(\boldsymbol{\sigma}\cdot\mathbf{n})_i = \sigma_{ij} n_j
\]

\begin{itemize}
  \item Traction vector on a surface
  \item Boundary term in momentum and energy balances
\end{itemize}
\end{frame}

% -------------------------------------------------
\begin{frame}{$\nabla \mathbf{v}$}
\[
(\nabla \mathbf{v})_{ij} = \frac{\partial v_i}{\partial x_j}
\]

\begin{itemize}
  \item Velocity gradient tensor
  \item Encodes strain rates and rotation
\end{itemize}
\end{frame}

% -------------------------------------------------
\begin{frame}{$\nabla^2 T$}
\[
\nabla^2 T = \frac{\partial^2 T}{\partial x_j \partial x_j}
\]

\begin{itemize}
  \item Scalar Laplacian
  \item Appears in heat conduction and diffusion
\end{itemize}
\end{frame}

% -------------------------------------------------
\begin{frame}{$\nabla^2 \mathbf{v}$}
\[
(\nabla^2 \mathbf{v})_i = \frac{\partial^2 v_i}{\partial x_j \partial x_j}
\]

\begin{itemize}
  \item Vector Laplacian
  \item Appears in viscous momentum transport
\end{itemize}
\end{frame}

% -------------------------------------------------
\begin{frame}{$\mathbf{v}\cdot\mathbf{b}$}
\[
\mathbf{v}\cdot\mathbf{b} = v_i b_i
\]

\begin{itemize}
  \item Power density of body forces
  \item Appears in the energy balance
\end{itemize}
\end{frame}

% =================================================
\section{Material Time Derivative}
% =================================================

% -------------------------------------------------
\begin{frame}{Eulerian and Lagrangian Descriptions}
\centering
\includegraphics[width=0.8\textwidth]{chap01_fig09_eulerian-lagrangian-rotating-drift_a.png}
\end{frame}

% -------------------------------------------------

\begin{frame}{Eulerian and Lagrangian Descriptions}
\centering
\includegraphics[width=0.8\textwidth]{chap01_fig09_eulerian-lagrangian-rotating-drift_b.png}
\end{frame}

% -------------------------------------------------
\begin{frame}{Material Time Derivative}
For a scalar field $f(\mathbf{x},t)$:
\[
\frac{Df}{Dt} = \frac{\partial f}{\partial t} + v_j \frac{\partial f}{\partial x_j}
\]

\begin{itemize}
  \item Time rate of change following a particle
  \item Combines local and convective effects
\end{itemize}
\end{frame}

% -------------------------------------------------
\begin{frame}{Material Derivative of Velocity}
\[
\frac{Dv_i}{Dt} = \frac{\partial v_i}{\partial t} + v_j \frac{\partial v_i}{\partial x_j}
\]

\begin{block}{Vector form}
\[
\frac{D\mathbf{v}}{Dt} = \frac{\partial \mathbf{v}}{\partial t} + (\mathbf{v}\cdot\nabla)\mathbf{v}
\]
\end{block}
\end{frame}

% =================================================
\section{Balance Laws}
% =================================================

% -------------------------------------------------

% -------------------------------------------------
\begin{frame}{Control Volume for Mass Balance}
\centering
\includegraphics[width=0.80\textwidth]{chap05_fig01_mass_potato.png}
\end{frame}

\begin{frame}{Mass Balance}
\[
\frac{\partial \rho}{\partial t} + \frac{\partial (\rho v_j)}{\partial x_j} = 0
\]
\vspace{0.5em}

\begin{itemize}
  \item Continuity equation
  \item Expresses conservation of mass
\end{itemize}
\end{frame}

% -------------------------------------------------

% -------------------------------------------------
\begin{frame}{Control Volume for Momentum Balance}
\centering
\includegraphics[width=0.80\textwidth]{chap05_fig03_potato_momentum.png}
\end{frame}

\begin{frame}{Momentum Balance}
\[
\rho \frac{Dv_i}{Dt} = \frac{\partial \sigma_{ij}}{\partial x_j} + \rho b_i
\]

\begin{itemize}
  \item Newton's second law per unit volume
  \item Stress divergence + body forces
\end{itemize}
\end{frame}

% -------------------------------------------------

% -------------------------------------------------
\begin{frame}{Control Volume for Energy Balance}
\centering
\includegraphics[width=0.80\textwidth]{chap05_fig05_potato_energy.png}
\end{frame}

\begin{frame}{Energy Balance}
\[
\rho \frac{De}{Dt} = \sigma_{ij}\frac{\partial v_i}{\partial x_j} - \frac{\partial q_j}{\partial x_j} + \rho S
\]

\begin{itemize}
  \item Internal energy form
  \item Mechanical power, heat flux, and sources
\end{itemize}
\end{frame}

% =================================================
\section{Advection--Diffusion Equation}
% =================================================

% -------------------------------------------------
\begin{frame}{Thermal Energy Balance}
Assuming:
\begin{itemize}
  \item Constant density and heat capacity
  \item Fourier's law: $q_i = -k\, \partial T/\partial x_i$
\end{itemize}

\[
\rho c \frac{DT}{Dt} = k \nabla^2 T + \rho S
\]
\end{frame}

% -------------------------------------------------
\begin{frame}{Expanded Form}
\[
\rho c \left( \frac{\partial T}{\partial t} + v_j \frac{\partial T}{\partial x_j} \right)
= k \frac{\partial^2 T}{\partial x_j \partial x_j} + \rho S
\]

\begin{itemize}
  \item Time-dependent advection--diffusion equation
\end{itemize}
\end{frame}

% -------------------------------------------------
\begin{frame}{Reduction to 1D}
Assume:
\begin{itemize}
  \item $v = v_x(x)$
  \item $T = T(x,t)$
\end{itemize}

\[
\rho c \left( \frac{\partial T}{\partial t} + v_x \frac{\partial T}{\partial x} \right)
= k \frac{\partial^2 T}{\partial x^2} + \rho S
\]
\end{frame}

% -------------------------------------------------
\begin{frame}{Why This Matters}
\begin{itemize}
  \item Clean bridge from continuum mechanics to numerics
  \item Every term maps directly to a discrete operator
  \item Sets the stage for FV, FD, FEM, and ML surrogates
\end{itemize}
\end{frame}

% -------------------------------------------------
\begin{frame}{Next Time}
\begin{itemize}
  \item Finite volume discretization of advection--diffusion
  \item Upwind vs.\ central differencing
  \item Stability and physical interpretation
\end{itemize}
\end{frame}

% -------------------------------------------------
\end{document}
