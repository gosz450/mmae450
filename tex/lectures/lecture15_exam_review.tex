\documentclass[aspectratio=169]{beamer}

\usetheme{Madrid}
\usepackage{amsmath, amssymb}
\usepackage{physics}

\title{MMAE 450 \\ Midterm 1 Review}
\author{Prof. Gosz}
\date{}

\begin{document}

%------------------------------------------------
\frame{\titlepage}

%------------------------------------------------
\begin{frame}{What This Exam Is Really About}
\Large
\begin{itemize}
    \item Taylor series
    \item Linearization (Newton)
    \item Discretization (finite difference)
    \item Stability
    \item Boundary conditions
\end{itemize}

\vspace{1em}
\centering
\Large
Everything reduces to:\\
\textbf{Linear systems or explicit updates}
\end{frame}

%------------------------------------------------
\section{Newton's Method}

\begin{frame}{Newton's Method (1 Variable)}

Solve:
\[
f(x) = 0
\]

Update:
\[
x_{k+1}
=
x_k
-
\frac{f(x_k)}{f'(x_k)}
\]

\vspace{1em}
\pause

\textbf{Derived from Taylor series.}
\end{frame}

%------------------------------------------------
\begin{frame}{Newton for Systems}

\[
\mathbf{R}(\mathbf{x}) = 0
\]

Linearize:
\[
\mathbf{J}(\mathbf{x}_k)\Delta \mathbf{x}
=
-
\mathbf{R}(\mathbf{x}_k)
\]

Update:
\[
\mathbf{x}_{k+1}
=
\mathbf{x}_k + \Delta \mathbf{x}
\]

\vspace{1em}
\pause

Residual → Jacobian → Linear solve → Update
\end{frame}

%------------------------------------------------
%------------------------------------------------
\section{Taylor Series}

\begin{frame}{Taylor Series (General Form)}

Start with:

\[
\Delta x = x - x_0
\]

First-order expansion:

\[
f(x)
\approx
f(x_0)
+
f'(x_0)\,\Delta x
\]

\pause

Second-order expansion:

\[
f(x)
=
f(x_0)
+
f'(x_0)\,\Delta x
+
\frac{1}{2} f''(x_0)\,(\Delta x)^2
+
O((\Delta x)^3)
\]

\end{frame}

%------------------------------------------------
\begin{frame}{Taylor Series for Finite Differences}

Choose:

\[
\Delta x = h
\quad \text{and} \quad
\Delta x = -h
\]

Then:

\[
f(x_0 + h)
=
f(x_0)
+
h f'(x_0)
+
\frac{h^2}{2} f''(x_0)
+
O(h^3)
\]

\[
f(x_0 - h)
=
f(x_0)
-
h f'(x_0)
+
\frac{h^2}{2} f''(x_0)
+
O(h^3)
\]

\end{frame}


%------------------------------------------------
\begin{frame}{Deriving the Second Derivative}

Add the two expansions:

\[
f(x_0 + h)
+
f(x_0 - h)
=
2f(x_0)
+
h^2 f''(x_0)
+
O(h^4)
\]

Rearrange:

\[
f''(x_0)
=
\frac{
f(x_0 + h)
-
2f(x_0)
+
f(x_0 - h)
}{h^2}
+
O(h^2)
\]

\vspace{1em}
\centering
Second-order accurate
\end{frame}











%------------------------------------------------
\section{1D Heat Conduction}

\begin{frame}{1D Steady-State Heat}

\[
\frac{d^2T}{dx^2} = 0
\]

Discretization:

\[
- T_{i-1}
+ 2T_i
- T_{i+1}
= 0
\]

\vspace{1em}
\centering
Tridiagonal linear system
\end{frame}

%------------------------------------------------
\begin{frame}{1D Transient — FTCS}

\[
T_i^{n+1}
=
T_i^n
+
r (T_{i+1}^n - 2T_i^n + T_{i-1}^n)
\]

\[
r = \frac{\alpha \Delta t}{\Delta x^2}
\]

\pause

Stability:
\[
r \le \frac{1}{2}
\]

\vspace{1em}
Explicit = conditionally stable
\end{frame}

%------------------------------------------------
\begin{frame}{1D Transient — Crank–Nicolson}

\[
A T^{n+1}
=
B T^n
\]

\pause

\begin{itemize}
\item Implicit
\item Unconditionally stable
\item Must solve linear system each step
\end{itemize}
\end{frame}


%===========================
%------------------------------------------------
\begin{frame}{Crank–Nicolson Example (3 Interior Nodes)}

\large
1D Heat Equation:
\[
\frac{\partial T}{\partial t}
=
\alpha \frac{\partial^2 T}{\partial x^2}
\]

\vspace{0.5em}

Domain: $x \in [0,1]$

\[
\Delta x = 0.25,
\qquad
\alpha = 1,
\qquad
r = \frac{\alpha \Delta t}{\Delta x^2} = 1
\]

\vspace{0.5em}

Dirichlet BCs:
\[
T_0 = 0,
\qquad
T_4 = 0
\]

\vspace{0.5em}

Initial interior values:
\[
\mathbf{T}^n
=
\begin{bmatrix}
0 \\ 1 \\ 0
\end{bmatrix}
\]

\vspace{1em}

\centering
\textbf{Compute } \mathbf{T}^{n+1}
\end{frame}
%=========================
%------------------------------------------------
\begin{frame}{Building the Matrix $A$}

Crank–Nicolson interior equation:

\[
-\frac{r}{2} T_{i-1}^{n+1}
+
(1+r) T_i^{n+1}
-
\frac{r}{2} T_{i+1}^{n+1}
=
\text{RHS}
\]

\vspace{0.8em}

For this example, $r=1$:

\[
-\frac{1}{2} T_{i-1}^{n+1}
+
2 T_i^{n+1}
-
\frac{1}{2} T_{i+1}^{n+1}
=
\text{RHS}
\]

\vspace{1em}

\centering
Each interior node contributes:

\[
\boxed{
\left[
-\tfrac{1}{2}
\quad
2
\quad
-\tfrac{1}{2}
\right]
}
\]

\end{frame}
%------------------------------------------------
\begin{frame}{Crank–Nicolson System}

For $r=1$, the discrete system becomes:

\[
\underbrace{
\begin{bmatrix}
2 & -\tfrac{1}{2} & 0 \\
-\tfrac{1}{2} & 2 & -\tfrac{1}{2} \\
0 & -\tfrac{1}{2} & 2
\end{bmatrix}
}_{A}
\begin{bmatrix}
T_1^{n+1} \\
T_2^{n+1} \\
T_3^{n+1}
\end{bmatrix}
=
\underbrace{
\begin{bmatrix}
\tfrac{1}{2} \\
0 \\
\tfrac{1}{2}
\end{bmatrix}
}_{\text{RHS}}
\]

\vspace{1.5em}

\end{frame}
%===================
%------------------------------------------------
\begin{frame}{Building the Right-Hand Side}

Crank–Nicolson equation:

\[
-\frac{r}{2} T_{i-1}^{n+1}
+
(1+r) T_i^{n+1}
-
\frac{r}{2} T_{i+1}^{n+1}
=
\frac{r}{2} T_{i-1}^{n}
+
(1-r) T_i^{n}
+
\frac{r}{2} T_{i+1}^{n}
\]

\vspace{0.8em}

For $r=1$:

\[
\text{RHS}
=
\frac{1}{2} T_{i-1}^{n}
+
0 \cdot T_i^{n}
+
\frac{1}{2} T_{i+1}^{n}
\]

\vspace{1em}

\centering
RHS depends only on known values at time level $n$.
\end{frame}
%====================
%------------------------------------------------
\begin{frame}{Compute the RHS Vector}

Given:
\[
\mathbf{T}^n =
\begin{bmatrix}
0 \\ 1 \\ 0
\end{bmatrix}
\]

Node $i=1$:
\[
\frac{1}{2} T_2^n = \frac{1}{2}
\]

Node $i=2$:
\[
\frac{1}{2}(T_1^n + T_3^n) = 0
\]

Node $i=3$:
\[
\frac{1}{2} T_2^n = \frac{1}{2}
\]

\[
\text{RHS}
=
\begin{bmatrix}
\frac{1}{2} \\
0 \\
\frac{1}{2}
\end{bmatrix}
\]

\end{frame}


%------------------------------------------------
\section{2D Heat Conduction}

\begin{frame}{2D Steady-State}

\[
T_{i,j}
=
\frac{1}{4}
\left(
T_{i+1,j}
+
T_{i-1,j}
+
T_{i,j+1}
+
T_{i,j-1}
\right)
\]

\vspace{1em}
\centering
5-point stencil
\end{frame}

%------------------------------------------------
\begin{frame}{2D Transient FTCS}

\[
T_{i,j}^{n+1}
=
T_{i,j}^n
+
r_x \Delta_x^2
+
r_y \Delta_y^2
\]

\pause

If $\Delta x = \Delta y$:

\[
r \le \frac{1}{4}
\]

\end{frame}

%------------------------------------------------
\section{Boundary Conditions}

\begin{frame}{Boundary Conditions}

\Large
\begin{tabular}{c c}
Dirichlet & Prescribed value \\
Neumann & Prescribed derivative \\
Robin & Mixed condition
\end{tabular}

\vspace{2em}

\centering
BCs modify discrete equations.
\end{frame}
%===========================




%------------------------------------------------
\begin{frame}{Closing Thought}

\Large
Taylor series → Discretization → Linear system

\vspace{1em}

Explicit scheme → Stability restriction

\vspace{1em}

Newton → Linearize → Solve linear system

\end{frame}

\end{document}