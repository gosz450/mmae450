%=======================================================================
% mmae450/tex/lectures/lecture_fvm_1D_intro.tex
% Standalone Beamer deck that inputs an existing Chapter 5 .tikz figure
% located in: mmae450/tex/figures/lectures/
%
% IMPORTANT:
% The .tikz file itself does: % ============================================================
% TikZ house styles for EPUB + PDF (font-stable, width-stable)
% ============================================================

\ifdefined\FigWidth\else
  \def\FigWidth{8.5cm}
\fi
\ifdefined\FigDesignWidth\else
  \def\FigDesignWidth{4.4}
\fi

\pgfmathsetlengthmacro{\u}{\FigWidth/\FigDesignWidth}

\tikzset{
  house/.style={
    x=\u, y=\u,
    line cap=round,
    line join=round,
    font=\figMainFont
  },
  % -------- Line weights --------
  axisLine/.style   ={line width=0.9pt},
  vectorLine/.style ={line width=1.2pt},
  auxLine/.style    ={line width=0.9pt, dashed},
  % -------- Arrowheads (single-headed) --------
  axisArrow/.style={
  axisLine,
  -{Stealth[length=3.2mm,width=2.2mm]},
  shorten >=1.2pt
},
vectorArrow/.style={
  vectorLine,
  -{Stealth[length=3.2mm,width=2.2mm]},
  shorten >=1.2pt
},
angleArrow/.style={
  axisLine,
  -{Stealth[length=2.6mm,width=1.8mm]},
  shorten >=1.0pt
},
  % -------- Distance / dimension arrows (double-headed) --------
  distArrow/.style={
    line width=0.8pt,
    {Stealth[length=2.6mm,width=1.8mm]}-{Stealth[length=2.6mm,width=1.8mm]},
    shorten >=1pt,
    shorten <=1pt
  },
  % -------- Label helpers --------
  axisLabel/.style   ={font=\figAxisFont},
  figureLabel/.style ={font=\figMainFont},
  % ---- Compatibility aliases ----
  axisTitle/.style = {axisLabel},
  tickLabel/.style = {figureLabel},
  mainText/.style  = {figureLabel}
}

% -------------------------------------------------
% Vector arrow style (for b, v, n, etc.)
% -------------------------------------------------
\tikzset{
  vec/.style={
    ->,
    thick,
    >=stealth
  }
}
% so we add the repo root to TeX's input search path using \input@path.
%=======================================================================

\documentclass[aspectratio=169]{beamer}

% --- Your beamer styling (adjust path/name if yours differs)
% -------------------------------------------------
% Packages
% -------------------------------------------------
\usepackage{amsmath, amsfonts}
\usepackage{graphicx}
\usepackage{booktabs}
\usepackage{hyperref}
\usepackage{bm}
\usepackage{siunitx}
\usepackage{listings}

% Figure search paths (relative to tex/lectures/)
\graphicspath{{../figures/lectures/}{../figures/shared/}}

% -------------------------------------------------
% Notes (speaker notes)
% -------------------------------------------------
\usepackage{pgfpages}
% Uncomment ONE of these for speaker notes:
% \setbeameroption{show notes} % notes only (for printing notes)
% \setbeameroption{show notes on second screen=right} % slides + notes

% -------------------------------------------------
% TikZ
% -------------------------------------------------
\usepackage{tikz}
\usetikzlibrary{matrix, calc}
\usepackage{xcolor} % (tikz loads xcolor, but explicit is fine)
\usetikzlibrary{arrows.meta, decorations.pathreplacing}


% -------------------------------------------------
% Themes
%--------------------------------------------------
\usetheme{default}
\usecolortheme{default}
\setbeamertemplate{navigation symbols}{}

% -----------------------------
% Code formatting
% -----------------------------
\definecolor{codegray}{RGB}{245,245,245}

\lstset{
  backgroundcolor=\color{codegray},
  basicstyle=\ttfamily\small,
  frame=single,
  breaklines=true,
  showstringspaces=false
}

% --- Shortcuts ---
%\newcommand{\dd}{\,\mathrm{d}}
\newcommand{\Grad}{\nabla}
\newcommand{\Div}{\nabla\!\cdot}
\newcommand{\bx}{\bm{x}}
\newcommand{\bn}{\bm{n}}
\newcommand{\bq}{\bm{q}}

%==========================
% Flipbook macro (put in preamble or before frames)
%==========================
\newcommand{\GSFlipFrame}[2]{%
\begin{frame}{Gauss--Seidel sweep (current point $(#1,#2)$)}
\centering
\hspace{3.5cm}
\vspace{3cm}
\begin{tikzpicture}[scale=2.2, every node/.style={font=\small}]
  % -----------------------------
  % USER SETTINGS
  % -----------------------------
  \def\Nx{6}   % i = 0..Nx-1
  \def\Ny{4}   % j = 0..Ny-1

  % current stencil location (interior)
  \def\ci{#1}
  \def\cj{#2}

  % Point spacing
  \def\s{0.85}

  % -----------------------------
  % STYLES
  % -----------------------------
  \tikzset{
    gpG/.style={circle, fill=green!70!black, inner sep=1.1pt},
    gpR/.style={circle, fill=red!75!black,   inner sep=1.1pt},
    gpbG/.style={circle, fill=green!70!black, inner sep=1.5pt},
    gpbR/.style={circle, fill=red!75!black,   inner sep=1.5pt},
    gpc/.style={circle, fill=blue, inner sep=1.7pt},
    stline/.style={dashed, thick},
    labij/.style={font=\normalsize},
    labp/.style={font=\scriptsize, text=gray!70},
  }

  % -----------------------------
  % OUTER RECTANGLE (domain)
  % -----------------------------
  \draw[thick] (0,0) rectangle ({(\Nx-1)*\s},{(\Ny-1)*\s});

  % -----------------------------
  % GRID POINTS + LABELS
  % -----------------------------
  \foreach \j in {0,...,\numexpr\Ny-1\relax} {
    \foreach \i in {0,...,\numexpr\Nx-1\relax} {

      \pgfmathsetmacro{\x}{\i*\s}
      \pgfmathsetmacro{\y}{\j*\s}

      % global index p = i + j*Nx (as in your first figure)
      \pgfmathtruncatemacro{\p}{\i + \j*\Nx}

      % boundary?
      \pgfmathtruncatemacro{\isB}{ (\i==0) || (\i==\Nx-1) || (\j==0) || (\j==\Ny-1) }
      % current center?
      \pgfmathtruncatemacro{\isC}{ (\i==\ci) && (\j==\cj) }

      % "already updated" (GS ordering): j<cj OR (j==cj and i<ci)
      \pgfmathtruncatemacro{\isUpdated}{ (\j<\cj) || ((\j==\cj) && (\i<\ci)) }

      % FORCE left boundary (i==0) and bottom boundary (j==0) to be green
      \pgfmathtruncatemacro{\forceGreen}{ \isB }

      % decide color state:
      % - center: blue
      % - else green if forced green OR updated
      % - else red
      \ifnum\isC=1
        \node[gpc] (P\i\j) at (\x,\y) {};
      \else
        \pgfmathtruncatemacro{\isGreen}{ (\forceGreen==1) || (\isUpdated==1) }
        \ifnum\isB=1
          \ifnum\isGreen=1
            \node[gpbG] (P\i\j) at (\x,\y) {};
          \else
            \node[gpbR] (P\i\j) at (\x,\y) {};
          \fi
        \else
          \ifnum\isGreen=1
            \node[gpG] (P\i\j) at (\x,\y) {};
          \else
            \node[gpR] (P\i\j) at (\x,\y) {};
          \fi
        \fi
      \fi

      % Labels (i,j) above-right; p below-right
      \node[labij, anchor=south west] at (\x+0.05,\y+0.05) {$(\i,\j)$};
      \node[labp,  anchor=north west] at (\x+0.05,\y-0.05) {$\p$};
    }
  }

  % -----------------------------
  % 5-POINT STENCIL (dashed)
  % -----------------------------
  \pgfmathsetmacro{\xc}{\ci*\s}
  \pgfmathsetmacro{\yc}{\cj*\s}

  \draw[stline] (\xc,\yc) -- ({(\ci+1)*\s},\yc);
  \draw[stline] (\xc,\yc) -- ({(\ci-1)*\s},\yc);
  \draw[stline] (\xc,\yc) -- (\xc,{(\cj+1)*\s});
  \draw[stline] (\xc,\yc) -- (\xc,{(\cj-1)*\s});

  % Mapping note
  %\node[font=\scriptsize, align=left] at ({(1)*\s+1.65},{(0)*\s-0.55})
  %{$p = i + j\,N_x$};
\end{tikzpicture}
\end{frame}%
}











% --- TikZ (beamer_styles.tex may already load it; harmless if duplicated)
\usepackage{tikz}
\usetikzlibrary{arrows.meta,calc,positioning,decorations.pathreplacing}

% --- Make sure TeX can resolve paths like tex/preamble/... used INSIDE the .tikz
% From mmae450/tex/lectures, the repo root is ../../
\makeatletter
\def\input@path{{../../}{../}{./}}
\makeatother

\title{Finite Volume Method in 1D}
\subtitle{Control Volume Discretization (Chapter 5 figure)}
\author{MMAE 450}
\date{}

\begin{document}

% ----------------------------------------------------------------------
\begin{frame}
  \titlepage
\end{frame}




% ----------------------------------------------------------------------
% ----------------------------------------------------------------------
% ----------------------------------------------------------------------
% ----------------------------------------------------------------------
\begin{frame}{Internal Energy Equation (with Source)}

After substituting the momentum equation into the total energy balance:

\[
\rho \frac{De}{Dt}
=
\nabla \cdot (k \nabla T)
+
\boldsymbol{\sigma} : \nabla \mathbf{v}
+
S
\]

\bigskip

Assume:
\begin{itemize}
\item negligible viscous dissipation
\item constant $\rho$, $k$, $c_p$
\item $e = c_p T$
\end{itemize}

\bigskip

When we use $e = c_p T$ and apply the definition of the material time derivative,

\[
\frac{D}{Dt}
=
\frac{\partial}{\partial t}
+
\mathbf{v}\cdot\nabla,
\]

the one-dimensional advection--diffusion equation can be written as

\[
\boxed{
\frac{\partial T}{\partial t}
+
v \frac{\partial T}{\partial x}
=
\alpha \frac{\partial^2 T}{\partial x^2}
+
\frac{S}{\rho c_p}
}
\]

where \(\alpha = \dfrac{k}{\rho c_p}\).

\end{frame}

% ----------------------------------------------------------------------
\begin{frame}{1D Finite Volume Discretization (Control Volume View)}
\centering

% If your .tikz uses \FigWidth internally, you can override it here:
\renewcommand{\FigWidth}{0.92\textwidth}

% Your figure lives here:
%=======================================================================
% chap05_fig08_1D_fv_discretization.tikz
% 1D finite volume discretization for advection–diffusion (HOUSE STYLE)
%=======================================================================

\providecommand{\FigWidth}{8.0cm}   % fallback only
\def\FigDesignWidth{6.2}           % includes label margins

% ============================================================
% TikZ house styles for EPUB + PDF (font-stable, width-stable)
% ============================================================

\ifdefined\FigWidth\else
  \def\FigWidth{8.5cm}
\fi
\ifdefined\FigDesignWidth\else
  \def\FigDesignWidth{4.4}
\fi

\pgfmathsetlengthmacro{\u}{\FigWidth/\FigDesignWidth}

\tikzset{
  house/.style={
    x=\u, y=\u,
    line cap=round,
    line join=round,
    font=\figMainFont
  },
  % -------- Line weights --------
  axisLine/.style   ={line width=0.9pt},
  vectorLine/.style ={line width=1.2pt},
  auxLine/.style    ={line width=0.9pt, dashed},
  % -------- Arrowheads (single-headed) --------
  axisArrow/.style={
  axisLine,
  -{Stealth[length=3.2mm,width=2.2mm]},
  shorten >=1.2pt
},
vectorArrow/.style={
  vectorLine,
  -{Stealth[length=3.2mm,width=2.2mm]},
  shorten >=1.2pt
},
angleArrow/.style={
  axisLine,
  -{Stealth[length=2.6mm,width=1.8mm]},
  shorten >=1.0pt
},
  % -------- Distance / dimension arrows (double-headed) --------
  distArrow/.style={
    line width=0.8pt,
    {Stealth[length=2.6mm,width=1.8mm]}-{Stealth[length=2.6mm,width=1.8mm]},
    shorten >=1pt,
    shorten <=1pt
  },
  % -------- Label helpers --------
  axisLabel/.style   ={font=\figAxisFont},
  figureLabel/.style ={font=\figMainFont},
  % ---- Compatibility aliases ----
  axisTitle/.style = {axisLabel},
  tickLabel/.style = {figureLabel},
  mainText/.style  = {figureLabel}
}

% -------------------------------------------------
% Vector arrow style (for b, v, n, etc.)
% -------------------------------------------------
\tikzset{
  vec/.style={
    ->,
    thick,
    >=stealth
  }
}

\begin{tikzpicture}[house]

  % ---- Styles ----
  \tikzset{
    halfMain/.style = {font=\figMainFont},
    halfAxis/.style = {font=\figAxisFont},
    axis/.style     = {axisLine, line width=0.9pt},
    face/.style     = {axisLine, line width=0.6pt},
    tick/.style     = {axisLine, line width=0.6pt},
    cellFill/.style = {fill=gray!15, draw=black, line width=0.8pt},
    brace/.style    = {decorate, decoration={brace, amplitude=4pt}, line width=0.6pt},
    flowArrow/.style= {axisArrow, line width=0.8pt},
    centerDot/.style= {fill=black},
  }

  % ---- Layout parameters (DESIGN UNITS) ----
  \def\L{5}          % total length in cell-width units
  \def\nCells{5}
  \pgfmathsetmacro{\dx}{\L/\nCells}

  % ---- Axis ----
  \draw[axis] (0,0) -- (\L,0);
  \node[halfAxis, below] at (0,-0.55) {$x=0$};
  \node[halfAxis, below] at (\L,-0.55) {$x=L$};

  % ---- Faces and centers ----
  \foreach \j in {0,...,\nCells} {
    \draw[face] (\j*\dx,0.35) -- (\j*\dx,-0.35);
  }
  \foreach \j in {1,...,\nCells} {
    \pgfmathsetmacro{\xc}{\j*\dx-0.5*\dx}
    \draw[tick] (\xc,0.18) -- (\xc,-0.18);
  }

  % ---- Highlight CV i (center cell) ----
  \pgfmathsetmacro{\i}{3}
  \pgfmathsetmacro{\xw}{\i*\dx-\dx}   % x_{i-1/2}
  \pgfmathsetmacro{\xe}{\i*\dx}       % x_{i+1/2}
  \pgfmathsetmacro{\xi}{(\xw+\xe)/2}  % x_i

  \path[cellFill] (\xw,-0.60) rectangle (\xe,0.60);

  % ---- Face labels centered above the highlighted faces ----
  \node[halfMain, above] at (\xw,0.62) {$x_{i-1/2}$};
  \node[halfMain, above] at (\xe,0.62) {$x_{i+1/2}$};

  % ---- Center dot + labels near it ----
  \fill[centerDot] (\xi,0) circle (1.6pt);

  \node[halfMain, anchor=west] at (\xi+0.00,0.14) {$x_i$};
  \node[halfMain, anchor=west] at (\xi+0.00,-0.18) {$T_i$};

  % ---- Δx brace (for the highlighted control volume) ----
  \draw[brace] (\xw,-1.00) -- node[halfMain, below=5pt] {$\Delta x$} (\xe,-1.00);

  % ---- Flow direction arrow ----
  \draw[flowArrow] (0,-1.20) -- ++(2.2,0)
    node[halfAxis, midway, below] {flow $v$};

\end{tikzpicture}

\end{frame}

% ----------------------------------------------------------------------
% ----------------------------------------------------------------------

% ----------------------------------------------------------------------
\begin{frame}{Finite Volume Formulation (No Source)}

Neglect the source term ($S=0$).

\bigskip

Write the 1D advection--diffusion equation in conservative form:
\[
\frac{\partial T}{\partial t}
+
\frac{\partial F}{\partial x}
=
0,
\qquad
F \equiv vT - \alpha \frac{\partial T}{\partial x}.
\]

\bigskip

Integrate over the control volume
$[x_{i-\tfrac12},\,x_{i+\tfrac12}]$:

\[
\int_{x_{i-\tfrac12}}^{x_{i+\tfrac12}}
\frac{\partial T}{\partial t}\,dx
+
\int_{x_{i-\tfrac12}}^{x_{i+\tfrac12}}
\frac{\partial F}{\partial x}\,dx
=
0.
\]

\end{frame}











\begin{frame}{Applying the Fundamental Theorem of Calculus}

Time derivative term:
\[
\int_{x_{i-\tfrac12}}^{x_{i+\tfrac12}}
\frac{\partial T}{\partial t}\,dx
=
\frac{d}{dt}
\int_{x_{i-\tfrac12}}^{x_{i+\tfrac12}} T\,dx.
\]

\bigskip

Flux term (FTC):
\[
\int_{x_{i-\tfrac12}}^{x_{i+\tfrac12}}
\frac{\partial F}{\partial x}\,dx
=
F(x_{i+\tfrac12}) - F(x_{i-\tfrac12}).
\]

\bigskip

Therefore,
\[
\boxed{
\frac{d}{dt}
\int_{x_{i-\tfrac12}}^{x_{i+\tfrac12}} T\,dx
=
F_{i-\tfrac12} - F_{i+\tfrac12}
}
\]

\bigskip

\centering
Storage = $\text{Influx} - \text{Outflux}

\end{frame}
% ----------------------------------------------------------------------
% ----------------------------------------------------------------------
\begin{frame}{Cell Average Definition}

From the integral balance:

\[
\frac{d}{dt}
\int_{x_{i-\tfrac12}}^{x_{i+\tfrac12}} T\,dx
=
F_{i-\tfrac12} - F_{i+\tfrac12}.
\]

\bigskip

Define the cell average temperature:

\[
T_i(t)
\;\equiv\;
\frac{1}{\Delta x}
\int_{x_{i-\tfrac12}}^{x_{i+\tfrac12}} T(x,t)\,dx.
\]

\bigskip

Therefore,

\[
\int_{x_{i-\tfrac12}}^{x_{i+\tfrac12}} T\,dx
=
T_i\,\Delta x.
\]

\bigskip

\centering
Unknown stored at the cell center represents a \textbf{cell average}.

\end{frame}
% ----------------------------------------------------------------------
% ----------------------------------------------------------------------
\begin{frame}{Semi-Discrete Finite Volume Equation}

Substitute the cell average into the conservation statement:

\[
\frac{d}{dt}
\left(
T_i \,\Delta x
\right)
=
F_{i-\tfrac12} - F_{i+\tfrac12}.
\]

\bigskip

Since $\Delta x$ is constant,

\[
\boxed{
\Delta x \,\frac{dT_i}{dt}
=
F_{i-\tfrac12}
-
F_{i+\tfrac12}
}
\]

\bigskip

Or equivalently,

\[
\frac{dT_i}{dt}
=
-\frac{1}{\Delta x}
\left(
F_{i+\tfrac12} - F_{i-\tfrac12}
\right).
\]

\bigskip

\centering
Storage = Influx − Outflux

\end{frame}

% ----------------------------------------------------------------------
% ----------------------------------------------------------------------
\begin{frame}{Face Flux: Diffusion Term}

Recall the total flux:
\[
F = vT - \alpha \frac{\partial T}{\partial x}.
\]

\bigskip

Diffusive contribution:
\[
F^{(d)} = -\alpha \frac{\partial T}{\partial x}.
\]

\bigskip

At the face $x_{i+\tfrac12}$, approximate the gradient using
a central difference:

\[
\left.
\frac{\partial T}{\partial x}
\right|_{i+\tfrac12}
\approx
\frac{T_{i+1}-T_i}{\Delta x}.
\]

\bigskip

Therefore,

\[
\boxed{
F^{(d)}_{i+\tfrac12}
=
-\alpha
\frac{T_{i+1}-T_i}{\Delta x}
}
\]

\end{frame}


% ----------------------------------------------------------------------
% ----------------------------------------------------------------------
% ----------------------------------------------------------------------
\begin{frame}{Face Flux: Convection Term (Central Approximation)}

Convective contribution:
\[
F^{(c)} = vT.
\]

\bigskip

At the face $x_{i+\tfrac12}$:
\[
F^{(c)}_{i+\tfrac12}
=
v\,T_{i+\tfrac12}.
\]

\bigskip

Using a central interpolation,

\[
T_{i+\tfrac12}
=
\frac{T_i + T_{i+1}}{2}.
\]

\bigskip

Therefore,

\[
\boxed{
F^{(c)}_{i+\tfrac12}
=
v \,\frac{T_i + T_{i+1}}{2}
}
\]

\end{frame}


% ----------------------------------------------------------------------
% ----------------------------------------------------------------------
\begin{frame}{Face Flux at $x_{i-\tfrac12}$ (Central Approximations)}

Total flux:
\[
F = vT - \alpha \frac{\partial T}{\partial x}.
\]

\bigskip

At the left face $x_{i-\tfrac12}$:

\[
T_{i-\tfrac12} \approx \frac{T_{i-1}+T_i}{2}
\qquad\Rightarrow\qquad
F^{(c)}_{i-\tfrac12}
=
v\,\frac{T_{i-1}+T_i}{2}.
\]

\[
\left.\frac{\partial T}{\partial x}\right|_{i-\tfrac12}
\approx
\frac{T_i-T_{i-1}}{\Delta x}
\qquad\Rightarrow\qquad
F^{(d)}_{i-\tfrac12}
=
-\alpha\,\frac{T_i-T_{i-1}}{\Delta x}.
\]

\bigskip

Therefore,
\[
\boxed{
F_{i-\tfrac12}
=
v\,\frac{T_{i-1}+T_i}{2}
-\alpha\,\frac{T_i-T_{i-1}}{\Delta x}
}
\]
\end{frame}
% ----------------------------------------------------------------------
% ----------------------------------------------------------------------
\begin{frame}{Semi-Discrete Update (Central Fluxes at Both Faces)}

Finite volume update:
\[
\frac{dT_i}{dt}
=
-\frac{1}{\Delta x}\left(F_{i+\tfrac12}-F_{i-\tfrac12}\right),
\qquad
F = vT - \alpha \frac{\partial T}{\partial x}.
\]

\bigskip

Using central face fluxes:

\[
F_{i+\tfrac12}
=
v\,\frac{T_i+T_{i+1}}{2}
-\alpha\,\frac{T_{i+1}-T_i}{\Delta x},
\qquad
F_{i-\tfrac12}
=
v\,\frac{T_{i-1}+T_i}{2}
-\alpha\,\frac{T_i-T_{i-1}}{\Delta x}.
\]

\bigskip

Substitute and simplify:

\[
\boxed{
\frac{dT_i}{dt}
=
-\;v\,\frac{T_{i+1}-T_{i-1}}{2\Delta x}
\;+\;
\alpha\,\frac{T_{i+1}-2T_i+T_{i-1}}{\Delta x^2}
}
\]

\bigskip

\centering
Central advection stencil + central diffusion stencil
\end{frame}

% ----------------------------------------------------------------------
\begin{frame}{Time Discretization (Forward Euler)}

Semi-discrete equation:
\[
\frac{dT_i}{dt}
=
-\;v\,\frac{T_{i+1}-T_{i-1}}{2\Delta x}
+\alpha\,\frac{T_{i+1}-2T_i+T_{i-1}}{\Delta x^2}.
\]

\bigskip

Approximate the time derivative at $t^n$ (Forward Euler):
\[
\frac{dT_i}{dt}\Big|_{t^n}\approx \frac{T_i^{n+1}-T_i^{n}}{\Delta t}.
\]

\bigskip

Therefore,
\[
\boxed{
T_i^{n+1}
=
T_i^{n}
-\frac{v\Delta t}{2\Delta x}\left(T_{i+1}^{n}-T_{i-1}^{n}\right)
+\frac{\alpha\Delta t}{\Delta x^2}\left(T_{i+1}^{n}-2T_i^{n}+T_{i-1}^{n}\right)
}
\]

\end{frame}

% ----------------------------------------------------------------------
% ----------------------------------------------------------------------
\begin{frame}{Cell Peclet Number and Scheme Behavior}

Recall the fully discrete update:

\[
T_i^{n+1}
=
T_i^{n}
-\frac{v\,\Delta t}{2\Delta x}\left(T_{i+1}^{n}-T_{i-1}^{n}\right)
+\frac{\alpha\,\Delta t}{\Delta x^2}\left(T_{i+1}^{n}-2T_i^{n}+T_{i-1}^{n}\right).
\]

\bigskip

Define the cell Peclet number:

\[
\boxed{
\mathrm{Pe}_{\Delta x}
=
\frac{v\,\Delta x}{\alpha}
}
\]

\bigskip

Interpretation:

\begin{itemize}
\item $\mathrm{Pe}_{\Delta x} \ll 1$ → diffusion-dominated
\item $\mathrm{Pe}_{\Delta x} \gg 1$ → advection-dominated
\item Central schemes may oscillate for large Peclet numbers
\end{itemize}

\bigskip

\centering
Motivation for upwind schemes
\end{frame}
% ----------------------------------------------------------------------

% ----------------------------------------------------------------------
\begin{frame}{Algorithm: 1D Advection--Diffusion (Central Scheme)}

Given:
\[
N,\;\Delta x,\;\Delta t,\;v,\;\alpha,
\qquad
T_i^0 \text{ for } i=0,\dots,N
\]

\bigskip

Define dimensionless parameters:
\[
\lambda_d = \frac{\alpha \Delta t}{\Delta x^2},
\qquad
\lambda_c = \frac{v \Delta t}{2\Delta x}.
\]

\bigskip

\textbf{For each time step $n \to n+1$:}

\begin{enumerate}
\item Apply boundary conditions to $T_0^n$ and $T_N^n$
\item For $i = 1,\dots,N-1$:
\[
T_i^{n+1}
=
T_i^{n}
-
\lambda_c\left(T_{i+1}^n - T_{i-1}^n\right)
+
\lambda_d\left(T_{i+1}^n - 2T_i^n + T_{i-1}^n\right)
\]
\item Advance to next time step
\end{enumerate}

\end{frame}
% ----------------------------------------------------------------------
\begin{frame}{Algorithm (IC, Sweep, BC Update)}

\textbf{Initialization:}
\begin{itemize}
\item Set initial condition: $T_i^0 = T(x_i,0)$ for $i=0,\dots,N$
\item Enforce boundary conditions on $T^0$ (set $T_0^0$, $T_N^0$)
\end{itemize}

\bigskip

\textbf{Time stepping: for $n=0,1,2,\dots$}
\begin{enumerate}
\item (BC at current time) Ensure $T_0^n$ and $T_N^n$ satisfy BCs
\item (Interior sweep) For $i=1,\dots,N-1$ compute
\[
T_i^{n+1}
=
T_i^{n}
-\lambda_c\left(T_{i+1}^n - T_{i-1}^n\right)
+\lambda_d\left(T_{i+1}^n - 2T_i^n + T_{i-1}^n\right)
\]
\item (BC at new time) Set $T_0^{n+1}$ and $T_N^{n+1}$ from BCs
\end{enumerate}

\end{frame}

\end{document}