\documentclass[aspectratio=169]{beamer}

% -------------------------------------------------
% Packages
% -------------------------------------------------
\usepackage{amsmath, amsfonts}
\usepackage{graphicx}
\usepackage{booktabs}
\usepackage{hyperref}
\usepackage{bm}
\usepackage{siunitx}
\usepackage{listings}

% Figure search paths (relative to tex/lectures/)
\graphicspath{{../figures/lectures/}{../figures/shared/}}

% -------------------------------------------------
% Notes (speaker notes)
% -------------------------------------------------
\usepackage{pgfpages}
% Uncomment ONE of these for speaker notes:
% \setbeameroption{show notes} % notes only (for printing notes)
% \setbeameroption{show notes on second screen=right} % slides + notes

% -------------------------------------------------
% TikZ
% -------------------------------------------------
\usepackage{tikz}
\usetikzlibrary{matrix, calc}
\usepackage{xcolor} % (tikz loads xcolor, but explicit is fine)
\usetikzlibrary{arrows.meta, decorations.pathreplacing}


% -------------------------------------------------
% Themes
%--------------------------------------------------
\usetheme{default}
\usecolortheme{default}
\setbeamertemplate{navigation symbols}{}

% -----------------------------
% Code formatting
% -----------------------------
\definecolor{codegray}{RGB}{245,245,245}

\lstset{
  backgroundcolor=\color{codegray},
  basicstyle=\ttfamily\small,
  frame=single,
  breaklines=true,
  showstringspaces=false
}

% --- Shortcuts ---
%\newcommand{\dd}{\,\mathrm{d}}
\newcommand{\Grad}{\nabla}
\newcommand{\Div}{\nabla\!\cdot}
\newcommand{\bx}{\bm{x}}
\newcommand{\bn}{\bm{n}}
\newcommand{\bq}{\bm{q}}

%==========================
% Flipbook macro (put in preamble or before frames)
%==========================
\newcommand{\GSFlipFrame}[2]{%
\begin{frame}{Gauss--Seidel sweep (current point $(#1,#2)$)}
\centering
\hspace{3.5cm}
\vspace{3cm}
\begin{tikzpicture}[scale=2.2, every node/.style={font=\small}]
  % -----------------------------
  % USER SETTINGS
  % -----------------------------
  \def\Nx{6}   % i = 0..Nx-1
  \def\Ny{4}   % j = 0..Ny-1

  % current stencil location (interior)
  \def\ci{#1}
  \def\cj{#2}

  % Point spacing
  \def\s{0.85}

  % -----------------------------
  % STYLES
  % -----------------------------
  \tikzset{
    gpG/.style={circle, fill=green!70!black, inner sep=1.1pt},
    gpR/.style={circle, fill=red!75!black,   inner sep=1.1pt},
    gpbG/.style={circle, fill=green!70!black, inner sep=1.5pt},
    gpbR/.style={circle, fill=red!75!black,   inner sep=1.5pt},
    gpc/.style={circle, fill=blue, inner sep=1.7pt},
    stline/.style={dashed, thick},
    labij/.style={font=\normalsize},
    labp/.style={font=\scriptsize, text=gray!70},
  }

  % -----------------------------
  % OUTER RECTANGLE (domain)
  % -----------------------------
  \draw[thick] (0,0) rectangle ({(\Nx-1)*\s},{(\Ny-1)*\s});

  % -----------------------------
  % GRID POINTS + LABELS
  % -----------------------------
  \foreach \j in {0,...,\numexpr\Ny-1\relax} {
    \foreach \i in {0,...,\numexpr\Nx-1\relax} {

      \pgfmathsetmacro{\x}{\i*\s}
      \pgfmathsetmacro{\y}{\j*\s}

      % global index p = i + j*Nx (as in your first figure)
      \pgfmathtruncatemacro{\p}{\i + \j*\Nx}

      % boundary?
      \pgfmathtruncatemacro{\isB}{ (\i==0) || (\i==\Nx-1) || (\j==0) || (\j==\Ny-1) }
      % current center?
      \pgfmathtruncatemacro{\isC}{ (\i==\ci) && (\j==\cj) }

      % "already updated" (GS ordering): j<cj OR (j==cj and i<ci)
      \pgfmathtruncatemacro{\isUpdated}{ (\j<\cj) || ((\j==\cj) && (\i<\ci)) }

      % FORCE left boundary (i==0) and bottom boundary (j==0) to be green
      \pgfmathtruncatemacro{\forceGreen}{ \isB }

      % decide color state:
      % - center: blue
      % - else green if forced green OR updated
      % - else red
      \ifnum\isC=1
        \node[gpc] (P\i\j) at (\x,\y) {};
      \else
        \pgfmathtruncatemacro{\isGreen}{ (\forceGreen==1) || (\isUpdated==1) }
        \ifnum\isB=1
          \ifnum\isGreen=1
            \node[gpbG] (P\i\j) at (\x,\y) {};
          \else
            \node[gpbR] (P\i\j) at (\x,\y) {};
          \fi
        \else
          \ifnum\isGreen=1
            \node[gpG] (P\i\j) at (\x,\y) {};
          \else
            \node[gpR] (P\i\j) at (\x,\y) {};
          \fi
        \fi
      \fi

      % Labels (i,j) above-right; p below-right
      \node[labij, anchor=south west] at (\x+0.05,\y+0.05) {$(\i,\j)$};
      \node[labp,  anchor=north west] at (\x+0.05,\y-0.05) {$\p$};
    }
  }

  % -----------------------------
  % 5-POINT STENCIL (dashed)
  % -----------------------------
  \pgfmathsetmacro{\xc}{\ci*\s}
  \pgfmathsetmacro{\yc}{\cj*\s}

  \draw[stline] (\xc,\yc) -- ({(\ci+1)*\s},\yc);
  \draw[stline] (\xc,\yc) -- ({(\ci-1)*\s},\yc);
  \draw[stline] (\xc,\yc) -- (\xc,{(\cj+1)*\s});
  \draw[stline] (\xc,\yc) -- (\xc,{(\cj-1)*\s});

  % Mapping note
  %\node[font=\scriptsize, align=left] at ({(1)*\s+1.65},{(0)*\s-0.55})
  %{$p = i + j\,N_x$};
\end{tikzpicture}
\end{frame}%
}











\title[MMAE 450]{MMAE 450 --- Computational Mechanics II}
\subtitle{Day 1: Course Introduction \& Computational Workflow}
\author{Mike Gosz}
\date{}

\begin{document}

% ---------------- Slide 1 ----------------
\begin{frame}
  \titlepage
  \note{
Welcome to MMAE 450. I’m Mike Gosz, and I’m excited to work with you this semester.
Today is about setting expectations and giving you the big picture: what computational mechanics is,
why it matters, and how we’ll work in this course.
If you leave today with a clear mental model of the workflow—model, equations, discretize, solve, interpret—you’re in great shape.
  }
\end{frame}

% ---------------- Slide 2 ----------------
\begin{frame}{Why Computational Mechanics?}

\begin{columns}[T]
  % ---- Left column: text ----
  \begin{column}{0.55\textwidth}
    \begin{itemize}
      \item Many engineering problems have \textbf{no closed-form solution}
      \item Computation bridges \textbf{theory} and \textbf{real systems}
      \item Central in:
      \begin{itemize}
        \item heat transfer
        \item solid mechanics and dynamics
        \item transport phenomena
      \end{itemize}
    \end{itemize}
  \end{column}

  % ---- Right column: figure ----
  \begin{column}{0.35\textwidth}
    \centering
    \includegraphics[width=\linewidth]{complex_mesh.png}
    \vspace{0.5em}

    \scriptsize
    Stress distribution in mechanical part with complex geometry.
  \end{column}
\end{columns}

\note{
This simple axial bar problem is something many of you have seen analytically.
It works nicely on paper because the stress and strain are uniform.
Later in the course, we’ll see how small changes—nonuniform geometry,
distributed loads, nonlinear material behavior—break closed-form solutions
and force us to rely on computational methods.
}
\end{frame}

% ---------------- Slide 3 ----------------
\begin{frame}{What Is Computational Mechanics?}

\begin{columns}[T] % top-aligned columns

% --------------------------------------------------
% LEFT COLUMN: text
% --------------------------------------------------
\begin{column}{0.55\textwidth}

\begin{enumerate}
  \item Physical system (assumptions + model)
  \item Governing equations
  \item Numerical approximation (discretization)
  \item Computational solution
  \item Interpretation and validation
\end{enumerate}

\vspace{0.5em}
\textbf{Key message:} Computational mechanics is about
\textbf{models}, not just numbers.

\end{column}

% --------------------------------------------------
% RIGHT COLUMN: figure
% --------------------------------------------------
\begin{column}{0.45\textwidth}
\centering
\includegraphics[width=\linewidth]{heat_potato.png}

\vspace{0.25em}
{\footnotesize
Complex geometry $\;\Rightarrow\;$ discretization $\;\Rightarrow\;$ computation
}
\end{column}

\end{columns}

\note{
This five-step loop is the backbone of the course.
You will see it again and again: start from physics, write equations,
discretize, solve, and then ask if the result makes sense.
The final step—interpretation and validation—is not optional.
If a plot looks wrong, we investigate why.
}

\end{frame}
% ---------------- Slide 4 ----------------
\begin{frame}{Where This Course Fits}

\begin{columns}[T]
  % ---- Left column: text ----
  \begin{column}{0.55\textwidth}
    \begin{itemize}
      \item Builds on:
      \begin{itemize}
        \item calculus
        \item differential equations
        \item linear algebra
      \end{itemize}
      \item Develops:
      \begin{itemize}
        \item numerical thinking
        \item computational implementation
        \item interpretation of results
      \end{itemize}
      \item Prepares you for:
      \begin{itemize}
        \item research and modeling
        \item engineering simulation tools
        \item machine learning models and cloud workflows
        \item data analytics
      \end{itemize}
    \end{itemize}
  \end{column}

  % ---- Right column: figure ----
  \begin{column}{0.45\textwidth}
    \centering
    \includegraphics[width=\linewidth]{chap01_fig04_three_surfaces.png}
    \vspace{0.5em}

    \scriptsize
    Intersection of nonlinear surfaces representing a system of equations.
  \end{column}
\end{columns}

\note{
This figure represents a system of nonlinear equations in three variables.
Each surface corresponds to one equation, and the solution is where all three
surfaces intersect.

Analytically, problems like this are very difficult or impossible to solve
in closed form. Computational methods—such as Newton’s method for systems—
allow us to approximate solutions and understand their behavior.

This is a good example of how the mathematics you already know leads naturally
to computational mechanics when systems become nonlinear or high-dimensional.
}
\end{frame}

% ---------------- Slide 5 ----------------
\begin{frame}{What This Course Emphasizes}
\begin{itemize}
  \item Modeling, not memorization
  \item Algorithms, not button-pushing
  \item Understanding:
  \begin{itemize}
    \item accuracy
    \item stability
    \item limitations and failure modes
  \end{itemize}
\end{itemize}
\vspace{0.5em}
\textbf{We will care why methods work---and when they fail.}
\note{
I’m not interested in you memorizing formulas without understanding what they mean.
We will learn the algorithms and the reasoning behind them.
A method that produces a number is not automatically correct—stability and accuracy matter.
If a method fails, that failure often teaches us something important about the physics or the numerics.
  }
\end{frame}

% ---------------- Slide 6 ----------------
\begin{frame}{Tools We Will Use}

\begin{columns}[T]

% --------------------------------------------------
% LEFT COLUMN: text
% --------------------------------------------------
\begin{column}{0.60\textwidth}

\begin{itemize}
  \item \textbf{Python} as the computational language
  \item \textbf{Jupyter notebooks} for code + math + narrative
  \item Core libraries:
  \begin{itemize}
    \item NumPy (arrays, linear algebra)
    \item SymPy (symbolic math, derivations)
    \item Matplotlib (plots and visualization)
    \item SciPy (numerical solvers, eigenproblems)
    \item Pandas (data handling and post-processing)
    \item scikit-learn (regression and surrogate models)
    \item PyTorch (physics-informed neural networks)
    \item Cloud workflows for scalable computation
  \end{itemize}
\end{itemize} % <-- add this
\vspace{0.5em}
\textbf{We will briefly review Python and notebooks to ensure everyone is aligned.}

\end{column}

% --------------------------------------------------
% RIGHT COLUMN: figure
% --------------------------------------------------
\begin{column}{0.40\textwidth}
\centering
\includegraphics[width=\linewidth]{notebook.png}

\vspace{0.25em}
{\footnotesize
Code, equations, and explanation in one place
}
\end{column}

\end{columns}

\note{
Python is our tool, not the subject of the course.
We’ll learn what we need as we go, and I’ll provide guided notebooks.
Your job is to run them, modify them, and develop intuition about what changes and why.
}

\end{frame}

% ---------------- Slide 7 ----------------
\begin{frame}{Why Jupyter Notebooks?}
\begin{itemize}
  \item Combine: formatted text, equations, executable code, and plots
  \item Ideal for:
  \begin{itemize}
    \item exploration
    \item verification of derivations
    \item clear communication of results
  \end{itemize}
\end{itemize}
\note{
A notebook is an engineering document: it shows what you did and why.
It’s not just code; it’s reasoning plus results.
Throughout the course you’ll use notebooks to explore, verify, and communicate your work.
  }
\end{frame}

% ---------------- Slide 8 ----------------
\begin{frame}{How the Course Is Structured}

\begin{columns}[T]

% --------------------------------------------------
% LEFT COLUMN: text (unchanged)
% --------------------------------------------------
\begin{column}{0.60\textwidth}

\begin{itemize}
  \item Weekly modules on Canvas
  \item Mix of:
  \begin{itemize}
    \item lectures
    \item guided notebooks
    \item homework assignments
  \end{itemize}
  \item Two midterms + final exam or final project
\end{itemize}

\end{column}

% --------------------------------------------------
% RIGHT COLUMN: figure pushed toward bottom
% --------------------------------------------------
\begin{column}{0.40\textwidth}
\centering

\vfill  % <-- pushes content down

\includegraphics[width=0.95\linewidth]{canvas.png}

\vspace{0.25em}
{\footnotesize
A consistent weekly rhythm: learn \(\rightarrow\) practice \(\rightarrow\) assess
}

\end{column}

\end{columns}

\note{
The rhythm of the course is consistent.
Each week you’ll have a module with materials and a notebook, and often a homework assignment.
We’ll have two midterms spaced through the term and then a final assessment or project at the end.
If you keep up week by week, the course becomes very manageable.
}

\end{frame}
% ---------------- Slide 10 ----------------
\begin{frame}{What You Should Do This Week}
\begin{itemize}
  \item Review the syllabus (PDF)
  \item Complete Module 0 setup (Python + environment)
  \item Run Notebook 01 and make small edits
  \item Start Homework 1 (reflection + small computations)
\end{itemize}
\note{
This first week is about getting comfortable with the workflow.
Run the notebook, change a few numbers, and see what happens.
Homework 1 is intentionally low pressure: it’s about computational thinking and interpretation.
If you hit setup issues, bring your laptop and we’ll get you unstuck early.
  }
\end{frame}
\end{document}