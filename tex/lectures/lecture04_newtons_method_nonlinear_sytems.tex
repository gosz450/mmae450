\documentclass{beamer}

% -------------------------------------------------
% Packages
% -------------------------------------------------
\usepackage{amsmath, amsfonts}
\usepackage{graphicx}
\usepackage{booktabs}
\usepackage{hyperref}
\usepackage{bm}
\usepackage{siunitx}
\usepackage{listings}

% Figure search paths (relative to tex/lectures/)
\graphicspath{{../figures/lectures/}{../figures/shared/}}

% -------------------------------------------------
% Notes (speaker notes)
% -------------------------------------------------
\usepackage{pgfpages}
% Uncomment ONE of these for speaker notes:
% \setbeameroption{show notes} % notes only (for printing notes)
% \setbeameroption{show notes on second screen=right} % slides + notes

% -------------------------------------------------
% TikZ
% -------------------------------------------------
\usepackage{tikz}
\usetikzlibrary{matrix, calc}
\usepackage{xcolor} % (tikz loads xcolor, but explicit is fine)
\usetikzlibrary{arrows.meta, decorations.pathreplacing}


% -------------------------------------------------
% Themes
%--------------------------------------------------
\usetheme{default}
\usecolortheme{default}
\setbeamertemplate{navigation symbols}{}

% -----------------------------
% Code formatting
% -----------------------------
\definecolor{codegray}{RGB}{245,245,245}

\lstset{
  backgroundcolor=\color{codegray},
  basicstyle=\ttfamily\small,
  frame=single,
  breaklines=true,
  showstringspaces=false
}

% --- Shortcuts ---
%\newcommand{\dd}{\,\mathrm{d}}
\newcommand{\Grad}{\nabla}
\newcommand{\Div}{\nabla\!\cdot}
\newcommand{\bx}{\bm{x}}
\newcommand{\bn}{\bm{n}}
\newcommand{\bq}{\bm{q}}

%==========================
% Flipbook macro (put in preamble or before frames)
%==========================
\newcommand{\GSFlipFrame}[2]{%
\begin{frame}{Gauss--Seidel sweep (current point $(#1,#2)$)}
\centering
\hspace{3.5cm}
\vspace{3cm}
\begin{tikzpicture}[scale=2.2, every node/.style={font=\small}]
  % -----------------------------
  % USER SETTINGS
  % -----------------------------
  \def\Nx{6}   % i = 0..Nx-1
  \def\Ny{4}   % j = 0..Ny-1

  % current stencil location (interior)
  \def\ci{#1}
  \def\cj{#2}

  % Point spacing
  \def\s{0.85}

  % -----------------------------
  % STYLES
  % -----------------------------
  \tikzset{
    gpG/.style={circle, fill=green!70!black, inner sep=1.1pt},
    gpR/.style={circle, fill=red!75!black,   inner sep=1.1pt},
    gpbG/.style={circle, fill=green!70!black, inner sep=1.5pt},
    gpbR/.style={circle, fill=red!75!black,   inner sep=1.5pt},
    gpc/.style={circle, fill=blue, inner sep=1.7pt},
    stline/.style={dashed, thick},
    labij/.style={font=\normalsize},
    labp/.style={font=\scriptsize, text=gray!70},
  }

  % -----------------------------
  % OUTER RECTANGLE (domain)
  % -----------------------------
  \draw[thick] (0,0) rectangle ({(\Nx-1)*\s},{(\Ny-1)*\s});

  % -----------------------------
  % GRID POINTS + LABELS
  % -----------------------------
  \foreach \j in {0,...,\numexpr\Ny-1\relax} {
    \foreach \i in {0,...,\numexpr\Nx-1\relax} {

      \pgfmathsetmacro{\x}{\i*\s}
      \pgfmathsetmacro{\y}{\j*\s}

      % global index p = i + j*Nx (as in your first figure)
      \pgfmathtruncatemacro{\p}{\i + \j*\Nx}

      % boundary?
      \pgfmathtruncatemacro{\isB}{ (\i==0) || (\i==\Nx-1) || (\j==0) || (\j==\Ny-1) }
      % current center?
      \pgfmathtruncatemacro{\isC}{ (\i==\ci) && (\j==\cj) }

      % "already updated" (GS ordering): j<cj OR (j==cj and i<ci)
      \pgfmathtruncatemacro{\isUpdated}{ (\j<\cj) || ((\j==\cj) && (\i<\ci)) }

      % FORCE left boundary (i==0) and bottom boundary (j==0) to be green
      \pgfmathtruncatemacro{\forceGreen}{ \isB }

      % decide color state:
      % - center: blue
      % - else green if forced green OR updated
      % - else red
      \ifnum\isC=1
        \node[gpc] (P\i\j) at (\x,\y) {};
      \else
        \pgfmathtruncatemacro{\isGreen}{ (\forceGreen==1) || (\isUpdated==1) }
        \ifnum\isB=1
          \ifnum\isGreen=1
            \node[gpbG] (P\i\j) at (\x,\y) {};
          \else
            \node[gpbR] (P\i\j) at (\x,\y) {};
          \fi
        \else
          \ifnum\isGreen=1
            \node[gpG] (P\i\j) at (\x,\y) {};
          \else
            \node[gpR] (P\i\j) at (\x,\y) {};
          \fi
        \fi
      \fi

      % Labels (i,j) above-right; p below-right
      \node[labij, anchor=south west] at (\x+0.05,\y+0.05) {$(\i,\j)$};
      \node[labp,  anchor=north west] at (\x+0.05,\y-0.05) {$\p$};
    }
  }

  % -----------------------------
  % 5-POINT STENCIL (dashed)
  % -----------------------------
  \pgfmathsetmacro{\xc}{\ci*\s}
  \pgfmathsetmacro{\yc}{\cj*\s}

  \draw[stline] (\xc,\yc) -- ({(\ci+1)*\s},\yc);
  \draw[stline] (\xc,\yc) -- ({(\ci-1)*\s},\yc);
  \draw[stline] (\xc,\yc) -- (\xc,{(\cj+1)*\s});
  \draw[stline] (\xc,\yc) -- (\xc,{(\cj-1)*\s});

  % Mapping note
  %\node[font=\scriptsize, align=left] at ({(1)*\s+1.65},{(0)*\s-0.55})
  %{$p = i + j\,N_x$};
\end{tikzpicture}
\end{frame}%
}











\title[MMAE 450 — Module 2]{Newton’s Method for Nonlinear Systems}
\subtitle{Module 2: Nonlinear Systems and Continuum Foundations}
\author{MMAE 450}
\date{}

\begin{document}

%-------------------------------------------------
\begin{frame}
  \titlepage
\end{frame}

%-------------------------------------------------
\begin{frame}{Why Nonlinear Solvers Matter}

\begin{columns}[T]
%-------------------------------------------------
\begin{column}{0.55\textwidth}
\begin{itemize}
  \item Most engineering models are \textbf{nonlinear}
  \item Linear systems appear \textbf{inside} nonlinear solvers
  \item Newton’s method underlies:
  \begin{itemize}
    \item nonlinear finite element analysis
    \item large-deformation mechanics
    \item computational fluid dynamics (implicit and steady solvers)
    \item multiphysics coupling (like fluid-structure interaction)
  \end{itemize}
\end{itemize}

\vspace{1em}
\centerline{\emph{Linear solvers are a subroutine.}}
\end{column}
%-------------------------------------------------
\begin{column}{0.45\textwidth}
\centering
\includegraphics[width=\textwidth]{flow_cylinder.png}

\vspace{0.5em}
{\scriptsize Example of a nonlinear CFD flow field}
\end{column}
%-------------------------------------------------
\end{columns}

\end{frame}
%-------------------------------------------------
\begin{frame}{Residual Formulation}
We seek the solution of a nonlinear system written as
\[
\bm{R}(\bm{u}) = \bm{0},
\]

where:
\begin{itemize}
  \item $\bm{u}$ is the vector of unknowns
  \item $\bm{R}$ is the nonlinear residual vector
\end{itemize}

\medskip
This formulation will appear repeatedly throughout the course.
\end{frame}

%-------------------------------------------------
\begin{frame}{Newton’s Method: Scalar Problem}
Given a nonlinear equation
\[
f(x) = 0,
\]
linearize about the current iterate $x^{(k)}$:
\[
f(x) \approx f(x^{(k)}) + f'(x^{(k)})(x - x^{(k)}).
\]

Setting the linear approximation to zero yields
\[
f'(x^{(k)})\,\Delta x = -f(x^{(k)}),
\qquad
x^{(k+1)} = x^{(k)} + \Delta x.
\]

\medskip
\textbf{Key idea:} Each Newton iteration solves a \textbf{linear equation}.
\end{frame}

%-------------------------------------------------
\begin{frame}{Newton’s Method (Scalar Algorithm)}
\textbf{Newton’s Method (Scalar)}

\begin{enumerate}
  \item Choose a convergence tolerance $\varepsilon > 0$
  \item Choose an initial guess $x^{(0)}$
  \item For $k = 0,1,2,\dots$:
  \begin{itemize}
    \item Evaluate the residual $f(x^{(k)})$
    \item Evaluate the derivative $f'(x^{(k)})$
    \item Solve $f'(x^{(k)})\,\Delta x = -f(x^{(k)})$
    \item Update $x^{(k+1)} = x^{(k)} + \Delta x$
    \item \textbf{Check convergence:} if $|f(x^{(k+1)})| < \varepsilon$, stop
  \end{itemize}
\end{enumerate}
\end{frame}

%-------------------------------------------------
\begin{frame}{Geometric Interpretation}

\begin{columns}[T]
%---------------------------------
\begin{column}{0.55\textwidth}
\begin{itemize}
  \item Newton’s method uses the tangent line at $x^{(k)}$
  \item The next iterate is where the tangent intersects the axis
  \item Convergence is rapid when the initial guess is good
  \item Poor initial guesses may lead to divergence
\end{itemize}
\end{column}
%---------------------------------
\begin{column}{0.45\textwidth}
\centering
\includegraphics[width=\textwidth]{chap01_fig03_newton_method.png}

\vspace{0.5em}
{\scriptsize Geometric interpretation of Newton’s method}
\end{column}
%---------------------------------
\end{columns}

\end{frame}
%-------------------------------------------------
\begin{frame}{Newton’s Method for Nonlinear Systems}
We now consider a system of nonlinear equations:
\[
\bm{R}(\bm{u}) =
\begin{bmatrix}
R_1(u_1,\dots,u_n)\\
\vdots\\
R_n(u_1,\dots,u_n)
\end{bmatrix}
= \bm{0}.
\]

Linearizing about $\bm{u}^{(k)}$ gives:
\[
\bm{R}(\bm{u})
\approx
\bm{R}(\bm{u}^{(k)})
+
\bm{J}(\bm{u}^{(k)})(\bm{u}-\bm{u}^{(k)}),
\]

where the Jacobian matrix is defined by
\[
J_{ij} = \frac{\partial R_i}{\partial u_j}.
\]

\medskip
\textbf{Interpretation:} $\bm{J}$ is the \textbf{tangent operator}.
\end{frame}

%-------------------------------------------------
\begin{frame}{Newton System, Update, and Convergence}

At iteration $k$, solve the linear system
\[
\bm{J}(\bm{u}^{(k)})\,\Delta\bm{u}
=
-\bm{R}(\bm{u}^{(k)}),
\]
and update
\[
\bm{u}^{(k+1)} = \bm{u}^{(k)} + \Delta\bm{u}.
\]

\medskip
\textbf{Newton’s Method (Systems)}

\begin{enumerate}
  \item Choose a convergence tolerance $\varepsilon > 0$
  \item Choose an initial guess $\bm{u}^{(0)}$
  \item For $k = 0,1,2,\dots$:
  \begin{itemize}
    \item Evaluate the residual $\bm{R}(\bm{u}^{(k)})$
    \item Assemble the Jacobian $\bm{J}(\bm{u}^{(k)})$
    \item Solve $\bm{J}(\bm{u}^{(k)})\,\Delta\bm{u} = -\bm{R}(\bm{u}^{(k)})$
    \item Update $\bm{u}^{(k+1)} = \bm{u}^{(k)} + \Delta\bm{u}$
    \item \textbf{Check convergence:} 
    if $\|\bm{R}(\bm{u}^{(k+1)})\| / \|\bm{R}(\bm{u}^{(0)})\| < \varepsilon$: \textbf{stop}
  \end{itemize}
\end{enumerate}

\end{frame}
%-------------------------------------------------
\begin{frame}{Summary}
\begin{itemize}
   \item Each iteration requires solving a linear system
  \item Exibits quadratic convergence when close to the solution
  \item Strong dependence on the initial guess
  \item Conditioning of the Jacobian affects robustness
  \item Failure modes are possible
\end{itemize}
\end{frame}

%-------------------------------------------------
\begin{frame}{Notebook Example}
\end{frame}
%-------------------------------------------------


%-------------------------------------------------
%-------------------------------------------------
\begin{frame}{Key Takeaways}
\begin{itemize}
  \item Newton’s method is a linearization engine
  \item The Jacobian is a tangent operator
  \item Linear solves dominate the computational cost
  \item This framework underlies the remainder of the course
\end{itemize}
\end{frame}

\end{document}