\documentclass[12pt]{article}
\usepackage{amsmath, amssymb}
\usepackage[margin=1in]{geometry}
\usepackage{enumitem}
\setlength{\parindent}{0pt}

\begin{document}

\begin{center}
\Large \textbf{MMAE 450 -- Midterm 1 \; Solution Key (Draft)} \\
\normalsize
\end{center}

%----------------------------------------------------------
\section*{Problem 1: Potential Energy $\rightarrow$ Residual $\rightarrow$ Newton (10 pts)}

\[
\Pi(u_1,u_2)
=
\frac{1}{4}u_1^{4}
+
\frac{1}{2}u_1^{2}
+
u_2^{2}
+
\frac{1}{2}u_1u_2
-
u_1
-
2u_2
\]
\[
\mathbf{f}(u_1,u_2)=\nabla \Pi(u_1,u_2)=\mathbf{0}
\]

\textbf{(a)} Correct choice: \fbox{A}

Compute:
\[
f_1=\frac{\partial \Pi}{\partial u_1}=u_1^3+u_1+\frac{1}{2}u_2-1,
\qquad
f_2=\frac{\partial \Pi}{\partial u_2}=2u_2+\frac{1}{2}u_1-2.
\]

\textbf{(b)} Correct choice: \fbox{A}
\[
\mathbf{J}=\frac{\partial \mathbf{f}}{\partial \mathbf{u}}
=
\begin{bmatrix}
3u_1^2+1 & \frac{1}{2}\\[2pt]
\frac{1}{2} & 2
\end{bmatrix}.
\]

\textbf{(c)} Newton step from $\mathbf{u}^{(0)}=[0\;\;0]^T$.

Evaluate:
\[
\mathbf{f}(\mathbf{u}^{(0)})=
\begin{bmatrix}-1\\-2\end{bmatrix},
\qquad
\mathbf{J}(\mathbf{u}^{(0)})=
\begin{bmatrix}
1 & \frac{1}{2}\\[2pt]
\frac{1}{2} & 2
\end{bmatrix}.
\]
Solve
\[
\mathbf{J}(\mathbf{u}^{(0)})\,\Delta\mathbf{u}=-\mathbf{f}(\mathbf{u}^{(0)})
=
\begin{bmatrix}1\\2\end{bmatrix}
\]
which gives
\[
\Delta\mathbf{u}=
\begin{bmatrix}
\frac{4}{7}\\[2pt]
\frac{6}{7}
\end{bmatrix}.
\]
Thus
\[
\mathbf{u}^{(1)}=
\begin{bmatrix}
\frac{4}{7}\\[2pt]
\frac{6}{7}
\end{bmatrix}.
\]

\textbf{(d)} Residual at $\mathbf{u}^{(1)}$:
\[
\mathbf{f}(\mathbf{u}^{(1)})=
\begin{bmatrix}
u_1^3+u_1+\frac{1}{2}u_2-1\\[2pt]
2u_2+\frac{1}{2}u_1-2
\end{bmatrix}_{(u_1,u_2)=(4/7,6/7)}
=
\begin{bmatrix}
\frac{64}{343}\\[4pt]
0
\end{bmatrix}.
\]

\textbf{(e)} Euclidean norm:
\[
\|\mathbf{f}(\mathbf{u}^{(1)})\|_2
=
\sqrt{\left(\frac{64}{343}\right)^2+0^2}
=
\frac{64}{343}
\approx 0.1866.
\]

%----------------------------------------------------------
\section*{Problem 2: 1D Steady Heat Conduction (10 pts)}

\[
\frac{d^2T}{dx^2}=0,\qquad T(0)=0,\; T(1)=100,\qquad \Delta x=0.25
\]
Nodes: $x_0=0,\;x_1=0.25,\;x_2=0.5,\;x_3=0.75,\;x_4=1$.

\textbf{(a)} Interior stencil:
\[
\frac{T_{i+1}-2T_i+T_{i-1}}{\Delta x^2}=0
\;\;\Rightarrow\;\;
-T_{i-1}+2T_i-T_{i+1}=0.
\]
Equations:
\[
\begin{aligned}
i=1:&\quad -T_0+2T_1-T_2=0 \Rightarrow 2T_1-T_2=0,\\
i=2:&\quad -T_1+2T_2-T_3=0,\\
i=3:&\quad -T_2+2T_3-T_4=0 \Rightarrow -T_2+2T_3=100.
\end{aligned}
\]

\textbf{(b)} Matrix system:
\[
\begin{bmatrix}
2 & -1 & 0\\
-1 & 2 & -1\\
0 & -1 & 2
\end{bmatrix}
\begin{bmatrix}
T_1\\T_2\\T_3
\end{bmatrix}
=
\begin{bmatrix}
0\\0\\100
\end{bmatrix}.
\]
Solution (linear profile):
\[
T_1=25,\quad T_2=50,\quad T_3=75.
\]

%----------------------------------------------------------
\section*{Problem 3: 1D Transient Heat -- FTCS (10 pts)}

\[
\frac{\partial T}{\partial t}=\alpha\frac{\partial^2T}{\partial x^2}
\]

\textbf{(a)} FTCS update:
\[
T_i^{n+1}=T_i^n+r\left(T_{i+1}^n-2T_i^n+T_{i-1}^n\right),
\qquad
r=\frac{\alpha\Delta t}{\Delta x^2}.
\]

\textbf{(b)} Definition and stability:
\[
r=\frac{\alpha\Delta t}{\Delta x^2},
\qquad
\text{stability requires } r\le \frac{1}{2}.
\]

\textbf{(c)} With $\alpha=1,\;\Delta x=0.1,\;\Delta t=0.01$:
\[
r=\frac{1\cdot 0.01}{(0.1)^2}=\frac{0.01}{0.01}=1
\]
Since $1>\frac{1}{2}$, the scheme is \fbox{unstable}.

%----------------------------------------------------------
\section*{Problem 4: Neumann BC at $x=L$ (Constant Heat Flux) (10 pts)}

Given:
\[
q^*=-k\left.\frac{\partial T}{\partial x}\right|_{x=L},\qquad x_N=L,\;\;x_{N-1}=L-\Delta x.
\]
Taylor about $x_N$:
\[
T(x)\approx T_N+(x-x_N)T'_N+\frac{1}{2}(x-x_N)^2T''_N.
\]

\textbf{(a)} Evaluate at $x_{N-1}=x_N-\Delta x$:
\[
T_{N-1}\approx T_N-\Delta x\,T'_N+\frac{\Delta x^2}{2}T''_N.
\]

\textbf{(b)} Use $T'_N=-\dfrac{q^*}{k}$:
\[
T_{N-1}\approx T_N+\Delta x\frac{q^*}{k}+\frac{\Delta x^2}{2}T''_N.
\]

\textbf{(c)} Solve for $T''_N$:
\[
T''_N\approx \frac{2}{\Delta x^2}\left(T_{N-1}-T_N-\Delta x\frac{q^*}{k}\right).
\]

\textbf{(d)} Boundary-node FTCS update:
\[
T_N^{n+1}
=
T_N^n+\alpha\Delta t\left(T''_N\right)^n
=
T_N^n+\alpha\Delta t\,
\frac{2}{\Delta x^2}\left(T_{N-1}^n-T_N^n-\Delta x\frac{q^*}{k}\right).
\]

%----------------------------------------------------------
\section*{Problem 5: Conceptual Questions (10 pts)}

\begin{enumerate}[label=\alph*)]

\item Crank--Nicolson is: \fbox{B}

\item Neumann BC specifies: \fbox{C}

\item Central differences are more accurate because: \fbox{B}

\item 5-point stencil uses: \fbox{C}

\item Pseudocode (one acceptable solution):
\begin{verbatim}
Given alpha, dx, dt
r = alpha*dt/dx^2
Initialize T[i] for i=0..N
Apply boundary conditions to T

for n = 0..Nt-1:
    for i = 1..N-1:
        Tnew[i] = T[i] + r*(T[i+1] - 2*T[i] + T[i-1])
    apply boundary conditions to Tnew
    T = Tnew
\end{verbatim}

\end{enumerate}

\end{document}