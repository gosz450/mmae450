\documentclass[12pt]{article}
\usepackage{amsmath, amssymb}
\usepackage[margin=1in]{geometry}
\usepackage{enumitem}
\setlength{\parindent}{0pt}

\begin{document}

\begin{center}
\Large
\textbf{MMAE 450 -- Midterm 1 (Makeup)} \\[6pt]
\normalsize
Time: 75 minutes
\end{center}

\vspace{0.5in}

\noindent
\textbf{Name:} \rule{4in}{0.4pt}

\vspace{0.3in}

\noindent
\textbf{Student ID:} \rule{3in}{0.4pt}

\vspace{0.5in}
\textbf{Instructions:}
\begin{itemize}
    \item Show all work for full credit.
    \item Clearly box final answers.
    \item Calculator and one 8.5$\times$11 formula sheet allowed.
    \item No phones or smart watches.
\end{itemize}

\newpage

%----------------------------------------------------------
\section*{Problem 1: Newton Method for a Nonlinear System (10 pts)}

Consider the nonlinear system

\[
\begin{cases}
f_1(u_1,u_2) = u_1^2 + 2u_1 - u_2 - 3 = 0 \\
f_2(u_1,u_2) = u_1 + u_2^2 - 2 = 0
\end{cases}
\]

\textbf{(a) (3 pts)}  
Compute the Jacobian matrix $\mathbf{J}(u_1,u_2)$.

\vspace{2in}

\textbf{(b) (5 pts)}  
Starting from

\[
\mathbf{u}^{(0)}=
\begin{bmatrix}
1\\1
\end{bmatrix},
\]

perform one full Newton iteration and compute $\mathbf{u}^{(1)}$.

\vspace{2in}

\textbf{(c) (2 pts)}  
Compute the residual norm $\|\mathbf{f}(\mathbf{u}^{(1)})\|$.

\newpage

%----------------------------------------------------------
\section*{Problem 2: 1D Steady Conduction with Source (10 pts)}

\[
\frac{d^2 T}{dx^2} + 5 = 0
\]

on $0 \le x \le 1$

with

\[
T(0)=0, \qquad T(1)=100.
\]

Use 3 interior nodes with $\Delta x = 0.25$.

\textbf{(a) (6 pts)}  
Write the finite difference equations at each interior node.

\vspace{2.5in}

\textbf{(b) (4 pts)}  
Write the matrix system $\mathbf{A}\mathbf{T}=\mathbf{b}$.

\newpage

%----------------------------------------------------------
\section*{Problem 3: FTCS Method (10 pts)}

\[
\frac{\partial T}{\partial t}
=
\alpha \frac{\partial^2 T}{\partial x^2}.
\]

\textbf{(a) (4 pts)}  
Derive the FTCS update formula for $T_i^{n+1}$.

\vspace{2in}

\textbf{(b) (3 pts)}  
Define the diffusion number

\[
r = \frac{\alpha \Delta t}{\Delta x^2}.
\]

State the stability condition.

\vspace{1.5in}

\textbf{(c) (3 pts)}  

If $\alpha = 1.5$, $\Delta x = 0.25$, and $\Delta t = 0.02$, determine whether the scheme is stable.

\newpage

%----------------------------------------------------------
%----------------------------------------------------------
\section*{Problem 4: Convection Boundary Condition (10 pts)}

Consider steady 1D conduction
\[
\frac{d^2 T}{dx^2}=0
\]

with a convection boundary condition at $x=0$:
\[
-k\frac{dT}{dx}(0) = h(T_0 - T_\infty),
\]
and $T(L)=100$.
\vspace{0.5in}

\textbf{(10 pts)}  
Using a forward difference approximation
\[
\frac{dT}{dx}(0) \approx \frac{T_1 - T_0}{\Delta x},
\]
derive the discrete boundary equation relating $T_0$ and $T_1$.


\newpage

%----------------------------------------------------------
\section*{Problem 5: Conceptual Understanding (10 pts)}

\textbf{(a) Multiple Choice (8 pts)}  
Select the correct answer for each question.

\begin{enumerate}[label=\arabic*.]

\item The Jacobian in Newton’s method represents:
\begin{itemize}
\item A) The solution vector
\item B) The matrix of partial derivatives of the residual
\item C) The inverse of the Hessian
\item D) The time derivative of the system
\end{itemize}

\item A Dirichlet boundary condition specifies:
\begin{itemize}
\item A) Heat flux
\item B) Temperature
\item C) Second derivative
\item D) Convection coefficient
\end{itemize}

\item The FTCS method becomes unstable when:
\begin{itemize}
\item A) $r < 1/2$
\item B) $r > 1/2$
\item C) $\Delta x \to 0$
\item D) $\alpha = 0$
\end{itemize}

\item The standard 2D Laplacian central difference uses:
\begin{itemize}
\item A) 3-point stencil
\item B) 4-point stencil
\item C) 5-point stencil
\item D) 9-point stencil
\end{itemize}

\end{enumerate}

\vspace{1em}

\textbf{(b) (2 pts)}  

Write clear pseudocode for advancing one time step of the 1D FTCS method.  
Include:
\begin{itemize}
\item Time loop
\item Interior node update
\item Boundary enforcement
\end{itemize}

\vspace{1.5in}

\vfill
\begin{center}
\textbf{End of Exam}
\end{center}

\end{document}