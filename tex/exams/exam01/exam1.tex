\documentclass[12pt]{article}
\usepackage{amsmath, amssymb}
\usepackage[margin=1in]{geometry}
\usepackage{enumitem}
\setlength{\parindent}{0pt}

\begin{document}

\begin{center}
\Large
\textbf{MMAE 450 -- Midterm 1} \\[6pt]
\normalsize
Time: 75 minutes
\end{center}

\vspace{0.5in}

\noindent
\textbf{Name:} \rule{4in}{0.4pt}

\vspace{0.3in}

\noindent
\textbf{Student ID:} \rule{3in}{0.4pt}

\vspace{0.5in}
\textbf{Instructions:}
\begin{itemize}
    \item Show all work for full credit.
    \item Clearly box or underline final answers.
    \item In addition to your laptop (Canvas under lockdown browser) you may use a calculator and one 8 1/2 by 11 formula sheet.
    \item No phones or smart watches.
\end{itemize}

\vspace{1em}

%----------------------------------------------------------
\section*{Problem 1: Potential Energy $\rightarrow$ Residual $\rightarrow$ Newton (10 pts)}

Let the total potential energy be a scalar function $
\Pi$  of the two displacement components $u_1$ and $u_2$:
\[
\Pi(u_1,u_2)
=
\frac{1}{4}u_1^{4}
+
\frac{1}{2}u_1^{2}
+
u_2^{2}
+
\frac{1}{2}u_1u_2
-
u_1
-
2u_2
\]
The equilibrium equations are obtained by minimizing $\Pi$, i.e.
\[
\mathbf{f}(u_1,u_2)=\boldsymbol{\nabla} \Pi(u_1,u_2)=\mathbf{0}.
\]

\textbf{(a) Multiple choice (2 pts)}  
Which is the residual vector $\mathbf{f}(u_1,u_2)$?

\begin{itemize}[label=$\square$]
\item A)
\(
\mathbf{f}=
\begin{bmatrix}
u_1^{3}+u_1+\frac{1}{2}u_2-1\\[2pt]
2u_2+\frac{1}{2}u_1-2
\end{bmatrix}
\)
\item B)
\(
\mathbf{f}=
\begin{bmatrix}
u_1^{3}+u_1+\frac{1}{2}u_2-2\\[2pt]
u_2+\frac{1}{2}u_1-2
\end{bmatrix}
\)
\item C)
\(
\mathbf{f}=
\begin{bmatrix}
u_1^{3}+u_1+u_2-1\\[2pt]
2u_2+u_1-2
\end{bmatrix}
\)
\item D)
\(
\mathbf{f}=
\begin{bmatrix}
3u_1^{2}+1\\[2pt]
2
\end{bmatrix}
\)
\end{itemize}

\vspace{0.75em}
\newpage
\textbf{(b) Multiple choice (2 pts)}  
Which is the Jacobian matrix $\mathbf{J}(u_1,u_2)=\dfrac{\partial \mathbf{f}}{\partial \mathbf{u}}$?

\begin{itemize}[label=$\square$]
\item A)
\(
\mathbf{J}=
\begin{bmatrix}
3u_1^{2}+1 & \frac{1}{2}\\[2pt]
\frac{1}{2} & 2
\end{bmatrix}
\)
\item B)
\(
\mathbf{J}=
\begin{bmatrix}
u_1^{3}+1 & \frac{1}{2}\\[2pt]
\frac{1}{2} & 2u_2
\end{bmatrix}
\)
\item C)
\(
\mathbf{J}=
\begin{bmatrix}
3u_1^{2}+1 & 1\\[2pt]
1 & 2
\end{bmatrix}
\)
\item D)
\(
\mathbf{J}=
\begin{bmatrix}
3u_1^{2}+1 & 0\\[2pt]
0 & 2
\end{bmatrix}
\)
\end{itemize}

\vspace{0.75em}

\textbf{(c) (4 pts)}  
Use one Newton iteration starting from
\[
\mathbf{u}^{(0)}=
\begin{bmatrix}
0\\0
\end{bmatrix}.
\]
That is, compute $\Delta \mathbf{u}$ from
\[
\mathbf{J}(\mathbf{u}^{(0)})\,\Delta \mathbf{u} = -\mathbf{f}(\mathbf{u}^{(0)}),
\]
then
\[
\mathbf{u}^{(1)} = \mathbf{u}^{(0)} + \Delta \mathbf{u}.
\]
Show your work.

\vspace{0.75in}

\textbf{(d) (1 pt)}  
Compute the residual vector $\mathbf{f}(\mathbf{u}^{(1)})$.

\vspace{1.5in}

\textbf{(e) (1 pt)}  
Compute the norm of the residual vector after one iteration:
\[
\|\mathbf{f}(\mathbf{u})^{(1)}\|.
\]

\vspace{1.25in}
\newpage

%----------------------------------------------------------
\section*{Problem 2: 1D Steady Heat Conduction (10 pts)}

Consider:
\[
\frac{d^2 T}{dx^2} = 0
\]
on $0 \le x \le 1$ with
\[
T(0)=0, \qquad T(1)=100.
\]

Use 3 interior nodes and $\Delta x = 0.25$.
\vspace{1.0in}

\textbf{(a) (5 pts)}  
Using a central difference approximation, write the discrete equations at each interior node.

\vspace{2in}

\textbf{(b) (5 pts)}  
Write the resulting linear system $\mathbf{A} \mathbf{T} = \mathbf{b}$ in matrix form.

\vspace{2in}
\newpage
%----------------------------------------------------------
\section*{Problem 3: 1D Transient Heat -- FTCS (10 pts)}

The 1D heat equation is:
\[
\frac{\partial T}{\partial t}
=
\alpha \frac{\partial^2 T}{\partial x^2}.
\]

\textbf{(a) (5 pts)}  
Write the FTCS update formula.

\vspace{1.5in}

\textbf{(b) (3 pts)}  
Define $r$ in terms of $\Delta x, \Delta t$ and $\alpha$ and state the stability condition.

\vspace{1.5in}

\textbf{(c) (2 pts)}  
If $\alpha=1$, $\Delta x=0.1$, and $\Delta t=0.01$, is the scheme stable? Show calculation.

\vspace{1.5in}

\newpage
%----------------------------------------------------------
\section*{Problem 4: Neumann BC at $x=L$ (Prescribed Constant Heat Flux) (10 pts)}

Consider 1D transient heat conduction
\[
\frac{\partial T}{\partial t}=\alpha\,\frac{\partial^2 T}{\partial x^2},
\]
on a uniform grid with spacing $\Delta x$. Let the right boundary be node $N$ at $x_N=L$, and the neighboring interior node be $N-1$ at $x_{N-1}=L-\Delta x$.

A constant heat flux is prescribed at $x=L$:
\[
q^* = -k\,\left.\frac{\partial T}{\partial x}\right|_{x=L},
\]
where $k$ is the thermal conductivity (a constant).

\vspace{0.4in}
Approximate the temperature field using a 3-term Taylor series about the boundary node $x_N=L$ as follows:
\[
T(x)\approx T_N + (x-x_N)\,T'_N + \frac{1}{2}(x-x_N)^2\,T''_N.
\]

\vspace{1.0in}

\textbf{(a) (2 pts)} Using the Taylor series above, evaluate $T(x)$ at $x_{N-1}=x_N-\Delta x$ to obtain an expression for $T_{N-1}$ in terms of $T_N$, $T'_N$, $T''_N$, and $\Delta x$.

\vspace{1.4in}

\textbf{(b) (2 pts)} Substitute the prescribed flux condition
\[
T'_N = \left.\frac{dT}{dx}\right|_{x=L} = -\frac{q^*}{k}
\]

into your expression from part (a).

\vspace{1.4in}

\textbf{(c) (2 pts)} From part (b), solve for the second derivative at the boundary node:
\[
T''_N=\left.\frac{d^2T}{dx^2}\right|_{x=L},
\]
in terms of $T_N$, $T_{N-1}$, $\Delta x$, $q^*$, and $k$.

\vspace{1.6in}

\textbf{(d) (4 pts)} For the 1D heat equation, the FTCS time update can be written as 
\[
T_N^{n+1} = T_N^n + \alpha\,\Delta t\,\left(T''_N\right)^n.
\]

Using your approximation for $(T_N'')^n$ obtained in part (c), write the explicit update formula for the boundary node $T_N^{n+1}$.

\newpage






%==========================================

\section*{Problem 5: Conceptual Questions (10 pts)}

\textbf{(Circle the correct answer.)}

\begin{enumerate}[label=\alph*)]

\item (2 pts) The Crank–Nicolson method for the heat equation is best described as:
\begin{itemize}
    \item A) Explicit and first-order accurate in time
    \item B) Implicit and unconditionally stable
    \item C) Explicit and unconditionally stable
    \item D) First-order accurate in time
\end{itemize}

\item (2 pts) A Neumann boundary condition specifies:
\begin{itemize}
    \item A) The temperature at the boundary
    \item B) The temperature at the previous time step
    \item C) The derivative of temperature (heat flux) at the boundary
    \item D) The interior node value
  \end{itemize}

\item (2 pts) The central difference approximation of the second derivative is more accurate than a one-sided approximation because:
\begin{itemize}
    \item A) It uses fewer grid points
    \item B) It has a smaller truncation error order
    \item C) It eliminates time discretization error
    \item D) It is unconditionally stable
\end{itemize}

\item (2 pts) The standard 5-point stencil for the 2D Laplacian at an interior node involves:
\begin{itemize}
    \item A) Only diagonal neighbors
    \item B) Only time-level neighbors
    \item C) The north, south, east, and west neighboring nodes
    \item D) Eight surrounding nodes
\end{itemize}



\item (2 pts) Write clear pseudocode for advancing the 1D heat equation in time using the FTCS method.





Your pseudocode should show:

- a time loop,\\
- the interior node update,\\
- and enforcement of boundary conditions.

\end{enumerate}
\newpage
\vfill
\begin{center}
\textbf{End of Exam}
\end{center}

\end{document}